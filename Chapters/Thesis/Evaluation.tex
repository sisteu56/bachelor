%************************************************
\chapter{Evaluation}\label{ch:evaluation}
%************************************************

This chapter evaluates the thesis core claims.
- for this, we investigate four projects

\section{Results}\label{sec:results}

\subsection*{\textbf{RQ1: Evaluation of Structural CFIs}}\label{sec:eval_struc_cfis}

Examining the structural CFIs of different projects can give us insights into the patterns of feature development and the usage of commits therin.
We first generally desribe our results with the help of figure \ref{fig:feature_sfbr_plot} and figure \ref{fig:commit_sfbr_plot} as well as the tables \ref{tab:feature_sfbr_table} and \ref{tab:commit_sfbr_table}.
We also explain particularly interesting results of individual commits and features by investigating their specific purpose and properties.

\subsubsection*{Patterns of Feature Development}\label{sec:eval_feature_development}

In the first part of RQ1, we examine the number of commits involved in the development of a feature and relate it to its size.
Figure \ref{fig:feature_sfbr_plot} illustrates our results in three different plots for each project.
Each row displays the results for one project with the name of the respective project being shown on the far left. 
In the first column, we show the number of structurally interacting commits for each feature in a bar-plot.
For all projects, we notice a wide distribution of structurally interacting commits between features.
Numerous features, across all projects, interact with less than 5 commits suggesting that they need very little work to be implemented in comparison to other features.
Both \textsc{xz} and \textsc{lrzip} have features structurally interacting with more than 20 commits, while \textsc{gzip} even has four features that interact with more than 60 commits.
With a mean of 29, the features of \textsc{gzip} have by far the highest number of interacting commits on average.
\textsc{xz}, \textsc{lrzip} and \textsc{bzip2} have much lower averages at 9, 16 and 4 respectively.
As all projects are of a compression domain, we can find several features with the same general functionality across the examined projects.
The \textsf{verbosity}-feature structurally interacts with the most commits in \textsc{xz} and \textsc{bzip2}, while the same is not the case for the \textsf{verbose}-feature of \textsc{gzip}.
Although \textsc{gzip} encompasses the highest average number of interacting commits, its \textsf{recursive}-feature only interacts with 2 commits, while the \textsf{recursive}-feature of \textsc{lrzip} interacts with 24 commits. \\
In the second column of figure \ref{fig:feature_sfbr_plot}, we display the calculated feature-sizes for each project.
Again, we see a wide range of different feature-sizes within all projects.
Most notabely, there are large jumps between adjacent features, for example from \textsf{to\_stdout} with a size of 450 to \textsf{no\_name} with a size of 7300 in \textsc{gzip}.
Similarly to the number of interacting commits, \textsc{gzip} has by far the highest average feature size at 2500.
The average feature sizes for the other projects range from 190 for \testc{xz} to 390 for \textsc{bzip2}. 
In table \ref{tab:feature_sfbr_table}, we display the average nesting degree of features for all projects.
Feature-nesting is the least common for \textsc{xz} and \textsc{lrzip} and most common for \textsc{gzip} with an average of 0.93 indicating that the majority of features are nested to large extents.
This is also the case for the four, by far largest, features of \textsc{gzip}, namely \textsf{no\_name, method, force} and \textsf{decompress}.
Specifically, 90 to 100\% of the instructions constituting their size are identical among the mentioned features.
It is therefore not surprising that most, particularly 66, of the commits interacting with each feature are identical for all four features.
For them, it is highly doubtful whether the size of a feature and the number of interacting commits reflect the actual number of \texttt{implementing} commits and instructions of a specific feature. \\
Finally, the metrics used in the two previously dicussed plots are compared to each other using the regression plot of figure \ref{fig:feature_sfbr_plot}.
That is, we compare the size of a feature with the number of commits that structurally interact with it.
The values of the blue dots each represent a feature and are therefore identical to the values of the bars in the two previous columns.
For both \textsc{xz} and \textsc{gzip}, we determine a strong positive correlation between the size of a feature and its number of structurally interacting commits.
Their linear correlation coeffecients are close to 1 at 0.91 and 0.978 respectively with p-values smaller than $10^{-4}$.
% This data provides strong evidence that our initial notion, that larger features generally need more commits to be implemented, is correct.
This data provides strong evidence that the observed correlation is statistically significant.
While \textsc{bzip2} and \textsc{lrzip} have positive correlation coefficients of 0.585 and 0.968 respectively, their p-vlaues are relatively high at over 0.1.
The high p-values can be explained by a lack of datapoints for \textsc{lrzip} and conflicting datapoints for \textsc{bzip2}.
For example, do both the \textsf{opMode}- and \textsf{srcMode}-feature of \textsc{bzip2} structurally interact with 6 commits, but encompass vastly different feature sizes.
Overall, the two projects do not produce conclusive statistical evidence that the size of a feature and the number of interacting commits is positively correlated.

\begin{table}[t]
\caption{Additional Information for Structural Analysis of Features}
\label{tab:feature_sfbr_table}
\centering
\begin{tabular}{l r}
\toprule
\textbf{Projects} & \overline{\textbf{Nesting Degree}} \\ 
\midrule
  xz    & 0.34 \\
  gzip  & 0.93 \\
  bzip2 & 0.64 \\
  lrzip & 0.37 \\
\bottomrule
\end{tabular}
\end{table}

\clearpage

\begin{figure}[htbp]
  \centering
  \includesvg[height=17cm]{gfx/results/RQ1/feature-sfbr-plot.svg}
  \caption{The Distribution of Feature Sizes and Structurally Interacting Commits of Features \\ and their Linear Correlation}
  \label{fig:feature_sfbr_plot}
\end{figure}

\clearpage

\begin{figure}[htbp]
  \centering
  \includesvg[height=14cm]{gfx/results/RQ1/commit-sfbr-plot.svg}
  \caption{The Distribution of the Number of Feature-Related Concerns for Commits}
  \label{fig:commit_sfbr_plot}
\end{figure}

\subsubsection*{Usage of Commits in Feature Development}\label{sec:eval_commit_usage}

As discussed in chapter \ref{sec:research_questions}, a best practice for commits in feature development is that one commit should generally change only one feature.
We investigate to what extent this holds in our investigated software projects by examining the number of feature-related concerns of a commit.
% since we determined this to be a fitting estimator for the number of features a commit changes or implements.
The distribution of how many commits have a certain number of concerns is displayed in figure \ref{fig:commit_sfbr_plot}.
\textsc{xz}, \textsc{gzip} and \textsc{lrzip} show similar distributions, which is why we discuss them now and \textsc{bzip2} later.
For \textsc{xz}, \textsc{gzip} and \textsc{lrzip}, we note that the majority of commits only have single feature-related concern.
The according percentages for commits with a single concern among all considered commits are 72\%, 69\% and 81\%.
The commit count gradually drops with an increasing number of concerns until it subsides to 0 at an x-value of 7, 6 and 4 respectively.
Interestingly, three commits of \textsc{gzip} have more than 10 concerns with the commit \textsf{33ae4134cc} even encompassing 47 concerns.
Said commit originally introduced the \textsf{gzip.c}-file containing the \textsl{main}-function among other things.
The exceptionally high number of \textsf{33ae4134cc}'s concerns can be explained by the fact that many of \textsc{gzip}'s 14 features are partially implemented within \textsf{gzip.c}.
Additionally, the code of features is heavily nested inside each other, which means that the instructions stemming from said code encompass many different combinations of feature regions.
For \textsc{bzip2}, there are much less commits used in feature-development compared to the other projects with only 9 commits part of structural CFIs.
The distribution consists of two clusters, the first being at a single concern and the second cluster being located between 7 and 10 concerns.
The second cluster encompassing 5 commits is slightly bigger than the first cluster with a commit count of 4.
Since we are only dealing with a few commits here, that still produce quite interesting and unique data, we have a closer look at the purpose of these commits and relate it to the number of their concerns.
We find that all commits part of the second cluster were used to publish new versions of \textsc{bzip2}, always encompassing over 1000 additions.
Three out of the four commits part of the first cluster introduced minor changes and bug fixes, never changing more than 30 source-code lines.
The remaining commit updated \textsc{bzip2} to version 1.0.4 fixing minor bugs from the previous version with many of the added source-code lines being used for comments. 

\begin{table}[t]
\caption{Additional Information for Structural Analysis of Commits}
\label{tab:commit_sfbr_table}
\centering
\begin{tabular}{l r r}
\toprule
\textbf{Projects} & \overline{\textbf{Concerns}} & \textbf{After Filtering Large Commits} \\ 
\midrule
  xz    & 1.48 & 1.29 \\
  gzip  & 2.40 & 1.55 \\
  bzip2 & 5.11 & 4.62 \\
  lrzip & 1.30 & 1.30 \\
\bottomrule
\end{tabular}
\end{table}

In table~\ref{tab:commit_sfbr_table}, we display the average number of feature-related concerns within a project both before and after filtering exceptionally large commits.
The first column contains the project to which the averages ​​in their respective row belong.
In the following column, we show the initial averages before applying the mentioned restriction.
\textsc{xz} and \textsc{lrzip} encompass the lowest average number of concerns at 1.48 and 1.3 respectively.
Although the distribution of \textsc{gzip} shown in figure \ref{fig:commit_sfbr_plot} is similar to those of the formerly mentioned projects, its initial average is comparatively high at 2.4.
The main cause for this are the discussed three commits of \textsc{gzip} with exceptionally many concerns.
After filtering large commits within a project, we note that the averages of \textsc{xz}, \textsc{lrzip} and \textsc{gzip} become more similar.
While the average slightly decreased for \textsc{xz} to 1.29, it stayed the same for \textsc{lrzip}.
The three dicussed commits of \textsc{gzip} were part of the filtered commits, which explains the large drop of its average to 1.55.
Unsurprisingly, the initial average of \textsc{bzip2} is by far the highest among all projects.
Since most of \textsc{bzip2}'s commits are relatively large, this is also the reason why they are not excluded from our analysis. 
Here, only one commit with 9 concerns is removed, which lowers the average to 4.58.
We also performed a qualitative analysis of the filtered commits to check whether our suspicion, that they were mostly used to refactor code, is true.
We find that many of them were used to import the project or newer versions to git.
The fact that such commits exist is unsurprising, considering that the investigated tools have been published before the release of git or before git was widespread.
While their purpose might not be refactoring code, it does make sense to filter these commits, as they are not accurately depicting the development process of the project itself or of its features.
Only two out of the eighteen commits we filtered in all projects, were actually used to refactor code, specifically moving contents between files or adapting the code to a newer version of C.
For half of all exceptionally large commits we found no reason, neither refactoring code nor importing the project to git, to be left out of our analysis.

\subsection*{\textbf{RQ2: Evaluation of Dataflow-based CFIs}}\label{sec:eval_df_cfis}

By investigating dataflow-based CFIs, we are able to determine the proportion of commits with dataflow interactions in a project as well as the number of commits that affect a feature through dataflow.
We factor in whether a dataflow interaction between a commit and a feature coincides with a structural interaction, to differentiate between dataflow stemming from outside or inside the regions of a feature.
This way, we aim to underline our thesis that commits inside of features are more likely to affect them trough dataflow, allowing us to especially focus on commits with an outside dataflow origin.
Our results regarding the first part of RQ2 are shown in figure \ref{fig:commit_dfbr_plot} and in the tables \ref{tab:commit_dfbr_table}, \ref{tab:commit_exclusive_and_partial_dfbr_table} and \ref{tab:commit_dfbr_rel_table}.
Figure \ref{fig:feature_dfbr_plot} displays our results for the features of a project and the commits that affect them through dataflow.

\begin{table}[t]
\caption{Additional Information for Dataflow Analysis of Commits}
\label{tab:commit_dfbr_table}
\centering
\begin{tabular}{l r r}
\toprule
\textbf{Projects} & \textbf{Active Commits} & \textbf{P($\rightsquigarrow$(C)\mid$\circledcirc$(C))} \\
\midrule
  xz    & 1039 & 0.883 \\
  bzip2 & 37 & 0.889 \\
  gzip  & 194 & 1.000 \\
\bottomrule
\end{tabular}
\end{table}

\subsubsection*{Proportion and Dependencies of Commits Affecting Features through Dataflow}\label{sec:eval_commit_dfbr}

At first, we examine the fraction of commits affecting features through dataflow among the \texttt{active} commits of a project.
There, we also evaluate the effects and dependencies of commits part of structural CFIs on said fraction.
Following this, we intend to determine the proportion of outside dataflow origins within the set of commits with dataflow interactions.
This also allows us to quantify the commits whose interactions with features can only be discovered through our dataflow analysis.
The number of active commits are shown in the second column of table \ref{tab:commit_dfbr_table} for each project.
We determined \textsc{xz} to have the highest number of active commits at 1039, while \textsc{bzip2} only has a tiny fraction of that at 37.
In the first plot of figure \ref{fig:commit_dfbr_plot}, the bars colored in red show what percentage of commits interact with features through dataflow.
We notice that the respective percentages are vastly different from project to project.
The majority of \textsc{gzip}'s active commits are part of dataflow-based CFIs at 62.9\%.
This percentage drops to about 27\% for \textsc{bzip2}, with another large reduction to 11.3\% for \textsc{xz}.
This means that more than every second commit in \textsc{gzip}, roughly every fourth commit in \textsc{bzip2} and every ninth commit in \textsc{xz} affects features through dataflow.
The bars colored in grey display the fraction of active commits structurally interacting with features.
In addition to the obvious fact of how often commits are used to implement features, this also bears important consequences for the fraction of commits with dataflow interactions.
\iffalse
The more commits are part of structural CFIs, the more of the code contributed by them and therefore the overall code of a project will be part of feature regions.
Logically, the accuracy of this estimation depends on many factors, such as the extend to which commits contributing code to features, also contribute code to other parts of a program.
Still, given the large disparity in the percentage of commits with structural interactions, we can be relatively certain that \textsc{xz} has the lowest proportion and \textsc{gzip} the largest proportion of feature-code with \textsc{bzip2} somewhere inbetween.
\fi
Comparing the two bar-types, we notice that the percentage of commits with structural interactions is lower than the percentage of commits with dataflow interactions for each project. \\
This phenomenon can be explained considering the values presented in the third column of table \ref{tab:commit_dfbr_table}.
There, we show the probablity for a commit to be part of any dataflow-based CFI, given that said commit is part of any structural CFI.
We see that the probabilities are slightly below 90\% for \textsc{xz} as well as \textsc{bzip2} and even at 100\% for \textsc{gzip}.
This means that by only taking into account commits part of structural CFIs, we already encounter a lot of commits that affect features through dataflow.
If we then consider the entire set of active commits, the number of commits with dataflow interactions will likely exceed the number of commits with structural interactions.
This also implies that most, if not all, commits constituting the grey bar of a project in turn constitute parts of the red bar as well. \\
We explore the topic of different dataflow interaction types more thoroughly with our second plot, which focuses on the \textsf{origin} of dataflow.
There, we have a closer look at the commits affecting features through dataflow and where their dataflow stems from.
In the second plot of figure \ref{fig:commit_dfbr_plot}, we separate the commits with dataflow interactions into three categories based on dataflow origin.
The first category, shown as the bar colored in blue, represents commits that only interact with features through outside dataflow.
They make up the majority of commits for \textsc{xz} at 56.4\%, but only 20\% as well as 25.4\% of commits for \textsc{bzip2} and \textsc{gzip} respectively.
The second category represents commits that interact with features through outside and inside dataflow.
Logically, these commits affect at least two features through dataflow and only structurally interact with a subset of them.
Surprsingly, they form the majority of commits for \textsc{bzip2} and \textsc{gzip} at 60 and 44.3\% respectively and around 20\% of commits for \textsc{xz}.
The proportion of commits that only interact with features through inside dataflow varies less between the projects.
\textsc{gzip} has the highest percentage at 31.3\%, \textsc{bzip2} the lowest at 20\% with \textsc{xz} inbetween the two at 23.9\%. \\
\iffalse
\begin{table}[t]
\caption{Commits exclusively/partially interacting with features \texttt{only} through dataflow}
\label{tab:commit_exclusive_and_partial_dfbr_table}
\centering
\begin{tabular}{lrrrr}}
    \toprule
    \textbf{Projects}
    \multirow{2}{*} &
      \multicolumn{2}{c}{\textbf{Commit Count}} &
      \multicolumn{2}{c}{\textbf{Percentage of all active commits}} \\
      & {Exclusive} & {Partial} & {Exclusive} & {Partial} \\
      \midrule
    xz & 66 & 23 & 6.4 & 2.2 \\
    bzip2 & 2 & 6 & 5.4 & 16.4 \\
    gzip & 31 & 54 & 16.0 & 27.9 \\
    \bottomrule
  \end{tabular}
\end{table}
\fi
Combining the data presented in both plots, we calculated the number and percentage of active commits in a project whose interactions with features were exclusively discovered by our dataflow analysis.
The respective values are shown in table \ref{tab:commit_exclusive_and_partial_dfbr_table} on the left-hand side of the second and third column.
With a count of 66, most commits were reveiled for \textsc{xz}, although its percentage remains comparatively small due to its high number of active commits.
\textsc{gzip} encompasses by far the highest percentage at 16\% with a commit count of 31.
On the right-hand side of the columns in table \ref{tab:commit_exclusive_and_partial_dfbr_table} we display the number and percentage of commits where only a subset of their interactions with features was discovered by our dataflow analysis.
These commits are most common in \textsc{gzip} with a count of 54 making up more than every fourth active commit at 27.9\%.
Similarly to the overall percentage of commits with dataflow interactions, they are less common in \textsc{bzip2} at 16.4\% and least common in \textsc{xz} at 2.2\%. \\
\begin{table}[t]
\caption{Relating Inside Dataflow to Outside Dataflow}
\label{tab:commit_dfbr_rel_table}
\centering
\begin{tabular}{l r r}
\toprule
\textbf{Projects} & \textbf{P(out$\rightsquigarrow$(C)\mid$\circledcirc$(C))} & \textbf{P(out$\rightsquigarrow$(C))} \\ 
\midrule
  xz    & 0.417 & 0.086 \\
  bzip2 & 0.667 & 0.216 \\
  gzip  & 0.596 & 0.381 \\
\bottomrule
\end{tabular}
\end{table}
% give a more accurate description of additional/new information our dataflow analysis can really add to CFIs in a project
% this is due to the fact that many dataflow interactions between a commit and a feature are implied/already reveiled due to them structurally interacting as well
We notice that commits interacting with features through inside dataflow, have a high chance to also interact with \texttt{other} features through outside dataflow.
By definition, all commits interacting with features through inside dataflow also structurally interact with them.
From previous explanations, we also know these commits are almost identical to the entire set of commits with structural interactions.
It follows that commits part of structural CFIs must also have a high, albeit slightly lower, chance to affect other features through outside dataflow.
We compare this probability to the probability of any active commit to interact with features through outside dataflow in table \ref{tab:commit_dfbr_rel_table}.
The strongest contrast occurs for \textsc{xz}, where 41.7\% of commits structurally interacting with features, interact with other features through outside dataflow, making them 4.8 times more likely to do so compared to any active commit of \textsc{xz} at 8.6\%.
The contrast is slightly weaker for \textsc{bzip2} with a 3.1 times higher likelihood and according percentages of 66.7 and 21.6\% respectively. 
\text{gzip} has the weakest contrast at 1.5 and respective probabilites of 59.6 and 38.1\%. 

\begin{figure}[htbp]
  \centering
  \includesvg[height=10cm]{gfx/results/RQ2/proportional_commit_dfbr_plot.svg}
  \caption{The Proportion of Commits with Dataflow Interactions and their Dataflow Origin}
  \label{fig:commit_dfbr_plot}
\end{figure}

\clearpage

\begin{figure}[htbp]
  \centering
  \includesvg[height=18cm]{gfx/results/RQ2/feature_dfbr_plot.svg}
  \caption{The Distribution and Proportion of Outside and Inside Commits Affecting a Feature through Dataflow and their Linear Correlation to the Size of a Feature}
  \label{fig:feature_dfbr_plot}
\end{figure}

\clearpage

\subsubsection*{Understanding Features and the Commits Affecting them Through Dataflow}\label{sec:eval_feature_dfbr}

When examining the commits affecting a feature through dataflow, we differentiate between those located outside and those, at least partially, located inside of the feature they interact with.
Investigating the ratio of outside and inside commits allows us to determine whether most dataflow interactions with commits occur inside or outside of the regions of a feature.
In a second evaluation step, we also relate the two metrics to the size of a feature.
Furthermore, we want to put our thesis, that the proportion of outside to inside commits of a feature decreases as its size increases, to the test.
Our results are shown in figure \ref{fig:feature_dfbr_plot} in three plots for each of our investigated projects.
In column one of the respective figure, we display the number of outside and inside commits for each feature in a bar plot.
We note that the majority of features in \textsc{xz}, \textsc{gzip} and \textsc{lrzip} are affected by more outside than inside commits through dataflow.
For \textsc{xz} and \textsc{gzip} this is underlined in median-values of 18.5 and 54 for outside, and 6.5 and 14 for inside commits, among their features.
% mention dichomitry between gzip's four largest features and the rest??
For features of \textsc{xz} and \textsc{gzip} with more outside than inside commits, we also notice very high ratios of outside to inside commits.
They range from 2 to 9 in \textsc{xz} with a mean of 4.4 and from 3 to 32 in \textsc{gzip} with a mean of 10.8.
For features with more inside than outside commits affecting them through dataflow, we determined much lower ratios and according ranges of inside to outside commits.
They range from 1.17 to 1.44 in \textsc{xz} with a mean of 1.3 and from 2.3 to 2.8 in \textsc{gzip} with a mean of 2.5.
At least for these two projects, the number of outside commits is often not only slightly higher than the number of inside commits, but instead exceeds it by many maginutes.
% - due to the high degree of feature-nesting determined for \textsc{gzip} in the evaluation of RQ1, we decided to test  
% same not true for inside commits
The features of \textsc{bzip2} exhibit a different general behavior than the features of the projects discussed above.
Specifically, 6 out of 8 features encompass more or at least the same number of inside than outside affecting them through dataflow.
The range and maximum number of interacting commits is also much lower compared to the other examined projects. \\
In the second column of figure \ref{fig:feature_dfbr_plot}, we compare the number of inside and the number of outside commits of a feature with its size in a regression plot.
As suspected during the formulation of RQ2, we find a strong positive correlation regarding the inside commits of features across all projects.
Especially, the correlation coefficients for \textsc{xz}, \textsc{gzip} and \textsc{lrzip} range close to 1 with a minimum value of 0.935.
\textsc{xz} and \textsc{gzip} encompass p-values of less than $10^{-4}$ supporting a statisitically significant correlation.
The correlations regarding the outside commits affecting features through dataflow are vastly different from project to project.
\textsc{xz} is the only project with a strong positive correlation with a coefficient of 0.824.
A low p-value of 0.003 outside of our rejection interval makes the data seem statistically significant at first.
However, the dots of its regression plot are scattered rather randomly and the number of interacting commits generally does not increase with rising feature-size.
We determined the \textsf{Verbosity}-feature to be the main reason for our calculated strong positive correlation.
This becomes clear when we disregard \textsf{Verbosity} in our analysis, which results in the coefficient dropping close to 0 and the p-value rising to 1.
\textsc{gzip} encompasses a negative correlation coefficient of -0.772 and a p-value of 0.002 justifiying the existence of significant statistical evidence.
The falling linear regression line is mainly formed by the two existing clusters, where one consists of four very large and the other cluster of nine relatively small features.
The former cluster consists of features that each interact with around 30 outside commits through dataflow.
The cluster consisting of features with small sizes encompasses higher values between 50 and 65.
The only feature of \textsc{gzip} that cannot be assigned to a cluster is the \textsf{recursive}-feature, which rather fits into the first cluster according to its number of outside commits and into the second cluster according to its size.
In \textsc{bzip2}, the number of outside commits affecting a feature through dataflow and its size are un-correlated.
A correlation coefficient of 0.545 and a high p-value of 0.633 do not provide significant statistical evidence for the existence of a correlation in \textsc{lrzip}. \\
In the third column of figure \ref{fig:feature_dfbr_plot}, we finally relate the size of a feature to its proportion of outside to inside commits.
At first, we notice that the calculated correlation coefficients are negative across all projects.
The fact that the coefficients of \textsc{xz} and \textsc{bzip2} are close to 0 and the p-values correspondingly high is illustrated by the rather random distribution of dots in their respective regression plots.
Their data should therefore not be regarded as statistically significant.
Compared to the aforementioned projects, \textsc{gzip}'s p-value of 0.035 is very low, while it also has a lower correlation coefficient of -0.588.
Similarly to \textsc{gzips}'s previous plots, we can identify two clusters.
A cluster of dots is again formed by the four largest features, encompassing proportions of outside to inside commits slightly below 0.5.
The other cluster is located at the beginning of the x-axis with a wide spread across the y-axis, indicating its low feature-size and a wide range of proportions from 3 to 32.
The coefficient of \textsc{lrzip} is by far the smallest among all projects at -0.99, but the small number of features results in a p-value of 0.088.
It follows that none of the determined negative correlations are statistically significant, as their p-values fall into our rejection interval of 97.5\%.

\subsection*{\textbf{RQ3: Evaluation of Author Interactions}}\label{sec:eval_author_interactions}

Here, we examine the authors structurally interacting with features, i.e. those involved in their development, as well as the authors introducing commits affecting features through outside dataflow.
As the sets of authors interacting with features structurally or through outside dataflow can overlap, we also investigate the number of authors that can only be discovered with our dataflow analysis.
The respective values for a feature's structural, dataflow and unique dataflow authors are shown in the first column of figure \ref{fig:author_cfi_plot} for each project.
In the second column of said figure, we also display our results regarding the comparison of the two metrics with the size of a feature. \\
The project \textsc{xz} has mostly been developed by a single author, which is reflected in our results.
As can be seen in figure \ref{fig:author_cfi_plot}, all except one feature have  been exclusively implemented by the same author.
We can conclude that this also is the developer affecting features through outside dataflow, since no feature has unique dataflow authors.
The features of \textsc{bzip2} mostly interact with one structural author and one or two dataflow authors.
Except for the \textsf{keepInputFiles}-feature, our dataflow analysis discovered one unique interacting author for each feature.
This changes for \textsc{lrzip}, where our dataflow analysis reveiled at least 2 and up to 5 unique authors for the three examined features.
Around half of the authors structurally interacting with a feature in \textsc{lrzip} also interact with it through outside dataflow.
Overall, the linear regressions relating the size of a feature to the number of its structural or dataflow authors show no significant correlation for the three mentioned projects. \\
\textsc{gzip}, on the other hand, provides us with much more comprehensive results.
In \textsc{gzip}'s bar-plot in figure \ref{fig:author_cfi_plot}, we can see that every feature has a large number of interacting authors.
The combined number of unique authors that interact with a feature in any way can be computed by the sum of its structural and its unique dataflow authors.
This means that the lowest number of interacting authors is 7 for the \textsf{recursive}-feature and the highest 14 for the \textsf{force}-feature.
On average, a feature has more dataflow than structural authors with respective averages of 9.1 and 5.7.
The average number of unique dataflow authors is 5.9, which means that our dataflow analysis discovers many additional interacting authors for the features in \textsc{gzip}.
The number of structurally interacting authors, i.e. authors implementing a feature, range from 2 to 13 between the features of \textsc{gzip}.
There is less variance in the number of a feature's outside dataflow authors ranging from 7 to 12.
Features with relatively few structural authors, have a comparatively high number of dataflow authors.
Features with many structural authors, starting from the \textsc{no\_name}-feature, have less outside dataflow authors.
At the same time, the number of their unique dataflow authors is comparatively low, implying that most of them also structurally interact with the according features.
The fact that the group of features with few structural authors has much more dataflow authors conditions them to have many unique dataflow authors.
Still, more than 80\% of their structural authors also interact with them through outside dataflow.
This is not the case for the group of features with many structural authors, where this is the case for less than 45\% of them.
Overall, there is a 57\% likelihood for the structural author of any feature to interact with the same feature through outside dataflow.
Similarly to the bar-plot of \textsc{gzip}, we notice two clusters when examining its regression plots in figure \ref{fig:author_cfi_plot}.
The first cluster is made up of features with few structural authors, whose respective sizes range between 20 and 450.
The second cluster consists of the four features who encompass more than 10 structural authors and much higher feature-sizes ranging from 7400 to 8200.
It is therefore unsurprising that we compute a strong positive correlation between the size of a feature and the number of its structurally interacting authors.
The respective correlation coefficient is 0.98, while the p-value is smaller than $10^{-4}$ providing evidence for a statistically significant correlation.
Regarding the dataflow authors of features, we note two similar clusters that consist of the same features of small and much larger sizes respectively.
Now, the smaller features have slightly more dataflow authors centering around 10 than the much larger features with an average of 5.
This shows itself in a strong negative correlation coefficient of -0.872 and a p-value of less than $10^{-4}$ falling out of our rejection interval of 97.5\%.

\begin{figure}[htbp]
  \centering
  \includesvg[height=18cm]{gfx/results/RQ3/author-cfi-plot.svg}
  \caption{The Distribution of Structural, Outside Dataflow and Unique Dataflow Authors of Features and their Linear Correlation to the Size of a Feature}
  \label{fig:author_cfi_plot}
\end{figure}

\clearpage 

\section{Discussion}\label{sec:discussion}

\subsection*{\textbf{RQ1: Structural CFIs}}\label{sec:disc_struc_cfis}

Generally, we have seen our expectations for possible results of RQ1 mostly confirmed.
Here, we briefly formulate our most important results, which we discuss in more detail in the following sub-sections.
Firstly, we found sufficient evidence to support the notion that the number of commits used to develop a feature is strongly positively correlated with the size of the code implementing its functionality. 
Secondly, we have high confidence that most commits, within our investigated projects, only changed or implemented functionality of one feature.

\subsubsection*{Patterns around Feature Development}

The features within the investigated projects exhibit a wide distribution of structurally interacting commits and sizes.
We consider this as a clear indication that the number of commits as well as the extent of source-code used to implement a feature varies a lot between them.
The range of least to most interacting commits and lowest to highest feature-size is hereby dependent on the specific project. \\
Determining the average nesting degree of features within a project showed us that, especially in the case of \textsc{gzip}, feature-nesting is more common than initially expected.
Most of the computed sizes of \textsc{gzip}'s respective features are made up of instructions that are part of multiple feature-regions besides its own. 
This makes it difficult to clearly interpret the data, as the number of instructions that actually implement a feature can differ greatly from our observed size. 
This phenonemon also bears consequences on they way commits are used in feature development and the CFIs resulting from this. 
If a developer wants to specifically change a feature nested inside other features, the according commit will necessarily produce structural CFIs with several features.  
This also casts some doubt how well the structural interactions of a feature with a high nesting degree accurately represent its development.
A concrete example of this are the four, by far largest, features of \textsc{gzip}.
In our evaluation of RQ1, we have already explained that they share the vast majority of instructions constituting their size, as well as the commits that structurally interact with them.
In these cases, it would be extremely helpful to be able to decide whether the respective instructions and commits are actually \texttt{implementing} commits and instructions of a particular feature.
While solving this problem goes beyond the scope of this work, we will adress the subject and possible solutions again in the \nameref{sec:futurework} section of this thesis. \\
We employed the pearson correlation coefficient to investigate the linear relationship between the size of a feature and the number of commits structurally interacting with it. 
While we found high correlation coefficients of 0.9 and above, the p-values of \textsc{lrzip} and \textsc{bzip2} were not low enough to justify the data to be statistically significant.  
We know that p-values are highly dependent on the number of datapoints, that is, the number of features of a project. 
Logically, projects with fewer features have less datapoints resulting in their p-values being quite high and falling into our rejection interval. 
With 10 and 14 features respectively, \textsc{xz} and \textsc{gzip} have by far the most datapoints and according p-values of less than $10^{-4}$ giving us most significant evidence for a strong positive correlation. 
Due to the above-mentioned problems caused by \textsc{gzip}'s high overall nesting degree, the significance of the found strong positive correlation is somewhat weakened.
On the other hand \textsc{xz} has a relatively low overall nesting degree, but still shows the same strong positive correlation.
Therefore, we are still able to confirm our original thesis of a strong dependency between the size of the implementing code and the number of implementing commits of a feature. \\
% - therefore, future research should prefer projects with at least 10 features so that correlations and other measurments produce trustful statistical data. \\
A possibility to circumvent the problem of too few datapoints could be to combine features of all projects into one big dataset.  
Here, we are faced with the issue that the way commits are used among projects might differ drastically. 
In our case, the number of instructions per structurally interacting commit of a feature varies greatly from project to project.  
The respective values are 17 for \textsc{xz}, 48 for \textsc{gzip}, 87 for \textsc{bzip2} and 13 for \textsc{lrzip}.  
This means that features of a similar size have, on average, vastly different numbers of structurally interacting commits across different projects. 
Upon these considerations, we are of the opinion that it makes more sense to test correlations within projects as we have done in this work. 

\subsubsection*{Usage of Commits in Feature Development}

In section \ref{sec:commit_concerns}, we explained that, due to feature-nesting, the number of features a commit structurally interacts with can be a faulty estimate of how many features a commit usually changes.
To achieve a more accurate measure, we introduced the concept of feature-related concerns of a commit (\ref{def:commit_concerns}).
There, we do not look at the structural CFIs of a commit separately, but rather focus on the entirety of them and the instructions they stem from.
As long as the CFIs occur in the same feature-related location, i.e. in instructions belonging to regions of the same set of features, we assume that only one specific feature was implemented by the commit.
A concrete example of why the feature-related concerns can be a more accurate estimate for the number of changed features is shown in listing \ref{lst:commit_concerns}.
% of course not 100\% accurate, for example when two features are implemented within the exact same code-space => we suspect this does not happen often 
Excluding \textsc{bzip2}, roughly $70\%$ of commits only have a single feature-related concern in the remaining projects.
Despite this, \textsc{gzip}'s average is quite high at 2.4, in comparison to \textsc{xz} and \textsc{lrzip} with 1.48 and 1.3 respectively.
It decreases to 1.55 after filtering large commits, such as commit \textsf{33ae4134cc} with an exceptionally high number of 47 feature-related concerns.
Its purpose was to import the original \textsf{gzip.c}-file into the repository, which justifies the exclusion of the commit.
Besides that, \textsc{gzip} only encompasses 14 features making it impossible for a commit to change 15 or more features.
The number of feature-related concerns being higher than the overall number of features also occurs for commits of \textsc{bzip2} and \textsc{lrzip}.
It follows that the concerns of commits can sometimes overestimate the number of changed features.
Naturally, this is most likely for commits structurally interacting with many features that also have high nesting degrees.
The code affected by these commits can include more combinations of different sets of feature regions than features present in the project.
For the respective commits, the number of feature-related concerns should rather be reduced to the total number of features they interact with. \\
To summarize our results, most commits of \textsc{xz}, \textsc{gzip} and \textsc{lrzip} have only one feature-related concern with an average of less than 1.6 after filtering large commits. 
Given that the feature-related concerns accurately reflect the number of changed features for most commits, we can confirm our initial thesis. 
Thus, the majority of commits that structurally interact with features are only used to change functionality of one feature. \\
In contrast to three projects discussed above, \textsc{bzip2} has shown to be a project whose results need to be treated more carefully.
Only 9 commits structurally interact with its features, encompassing much less datapoints compared to the other investigated projects.
6 out of 9 commits were used to import the initial project to git and iteratively update the code to newer versions.
These commits are not exemplary of the common development process of a project and its features in a git repository.
The actual development of the tool in the git repository began after \textsc{bzip2} was finally updated to version 1.0.6.
The remaining 3 commits introduce minor changes and small bug fixes to the respective features they interact with.
This means that \textsc{bzip2}'s features were largely already implemented when the actual development within the repository started.
Thus, the average number of concerns of a commit as well as their overall distribution are not good representatives of other projects developed in git. 

\subsection*{\textbf{RQ2: Dataflow-based CFIs}}\label{sec:disc_df_cfis}

We were able to confirm that structural CFIs heavily coincide data flow-based CFIs.
This allowed us to formulate a lower bound for the proportion of commits with dataflow interactions, namely the proportion of commits with structural interactions.
While there is a lot of variation in the proportion of commits affecting features through dataflow within the investigated projects, said proportion heavily depends on a project's extent of feature-code and the number of commits implementing said code.
Furthermore, our dataflow analysis uncovered a significant amount of interactions between commits and features that we could not find with our previous structural analysis.
We were also able to uncover that commits that already touch feature-code have an increased probability of influencing other features via outside dataflow.
We transferred this knowledge to features and the commits affecting them through dataflow by determining that many of their outside dataflow commits are also part of other features.
In general, the majority of features are influenced by outside commits, whereas this dataflow often originates from other features.
We found no statistically significant correlation between the number of outside commits of a feature and its size.
This also applies to the proportion of outside to inside commits, although we suspect that more data points and examined projects could lead to more substantive evidence here.

\subsubsection*{Proportion and Dependencies of Commits Affecting Features through Dataflow}

We have seen that the proportion of commits part of dataflow-based CFIs varies a lot between our investigated projects.
While only 11\% of commits in \textsc{xz} affect features through dataflow, the majority of commits do so in \textsc{gzip}.
One major reason for the discussed differences is the fact that the proportion of structural commits acts as the \texttt{lowest bound} for the proportion of commits with dataflow interactions.
As 90-100\% of commits part of structural CFIs are also part of dataflow-based CFIs, the percentage of commits part of dataflow-based CFIs never falls below the percentage of commits part of structural interactions.
The effects of said lowest bound can be recognized in the first plot of figure \ref{fig:commit_dfbr_plot}, where a project's rank among the dataflow proportions of commits reflects its rank among the structural proportions.
That is also because the percentages of commits, whose interactions with features were only discovered by our dataflow analysis, are the same or only a fraction of the percentages of those with structural interactions.
Similar strong differences in the structural proportions of commits can thus also be found in the dataflow proportions.
Including structural CFIs in the evaluation of our dataflow-based CFIs was definitely necessary to properly categorize their proportion and the new information they could add to the analysis of a project.
A substantial share of the examined dataflow interactions stem from our previously investigated set of structural interactions.
As we have determined structural CFIs to heavily coincide with dataflow-based CFIs, we already know of this set of dataflow interactions prior to our dataflow analysis. \\
Nevertheless, we believe that our dataflow analysis is very important to fully assess commit-feature interactions and their frequency within a project.
For \textsec{xz}, the majority of commits interacting with features were exclusivly reveiled by our dataflow analysis.
For 8, 21 and 44\% of active commits in \textsc{xz}, \textsc{bzip2} and \textsc{gzip}, it reveiled additional interactions with features that we could not discover with our structural analysis.
This means that the data of, on first sight unrelated, commits and features can be highly connected. 
Besides that, we found that many commits interacting with features both structurally and through dataflow, also interact with \texttt{other} features \texttt{only} through dataflow.
This lead us to examine whether commits structurally interacting with features are more likely to affect features through outside dataflow in comparison to any active commit.
We found sufficient evidence to support the notion that this is indeed the case, including an almost 5-times higher probability in \textsc{xz}.
This suggests that data, changed or allocated by a commit in one feature, is often used within a different feature again.
% Of course, it can also happen that a commit changes code outside of features in addition to feature code and the data flow into other features originates exclusively from the latter code.
% With our current analysis in VaRA, we cannot determine the number of these occurrences and therefore exclude them from our analysis.
% However, we see little reason to assume that they influence our results to a statistically significant extent. 
% In RQ1, we saw that most commits only change one feature, so there is a strong suspicion that most commits that touch feature code only touch feature code.
Therefore, a developer changing code of a feature should be aware that data of said code has an increased probability to affect the functionality of other features.
This also provides evidence that feature-code and remaining code of a program is more separate from each other than the initial percentage of commits with outside dataflow interactions would suggest.
When discussing the second part of RQ2, we will seek to underline this notion by determining how often outside commits affecting a feature through dataflow, structurally interact with other features. \\
Lastly, we discuss the extent of feature-code in a project and its consequences on the proportion of commits with structural as well as dataflow interactions.
In RQ1, we have confirmed that a feature's number of structurally interacting commits is strongly positively correlated with its size for \textsc{xz} and \textsc{gzip}.
It follows that a higher extent of feature-code implies a higher proportion of commits with structural interactions.
As the latter proportion acts as a lower bound for the proportion of commits with dataflow interactions, we assume the exent of feature-code to also have a strong influence on said proportion.
At the same time, less feature-code and a higher number of active commits could elavate the share of commits affecting features through outside dataflow.
This is due to a lower share of structurally interacting commits and therefore less inside dataflow as well as a higher number of active commits located outside of feature-regions leading to more possible outside dataflow.
We decided to compute the extent of feature-code in a project as the fraction of all instructions that are part of feature regions.
Thus, we determined the extent of feature-code to be 17.2\% in \textsc{xz} and 57.6\% in \textsc{gzip}. 
In our evaluation we have also seen that \textsc{xz} encompasses five times as many active commits than \textsc{gzip}.
We find that our observations in RQ2 roughly match these values considering our previous explanations.
Accordingly, \textsc{xz} has a higher share of commits affecting features through outside dataflow but a substantially lower share of active commits part of dataflow interactions.
It should be noted however that the exent of feature-code does not have an exact relation to the proportion of commits with structural interactions.
From \textsc{xz}'s 17\% extent of feature-code, we would expect a signficantly higher number than the 5.8\% of commits part of structural CFIs for example.

\subsubsection*{Understanding Features and the Commits Affecting them Through Dataflow}

Especially in the case of \textsc{xz} and \textsc{gzip}, we have found significant evidence to support the notion that features are generally affected by more outside than inside commits through dataflow.
This means that the majority of commits affecting a feature through dataflow are located outside of the feature they interact with.
Thus, most of the data belonging to the detected interactions of features originates outside of their regions and subsequently flows into them.
The ratio of outside to inside commits is often extremely skewed towards outside commits, while the opposite is rarely the case and if so, to a much smaller degree.
One reason for this is that even features with little size are affected by many outside commits, but only few inside commits.
That is because the number of a feature's inside commits is strongly positively correlated with its size, while the same is not the case for a feature's outside commits.
Specifically, we did not find evidence to suggest that the size of a feature and the number of outside commits affecting it through dataflow are correlated in some way or another.
The positive correlation found for \textsc{xz} is mainly driven by its largest feature, where the correlation coefficient subsides to 0 without it.
\textsc{gzip} shows a statistically signficant negative correlation, \textsc{bzip2} no correlation and \textsc{lrzip} a statisitcally insignifcant positive correlation.
From the previous explanations, it follows that the number of inside commits generally increases with rising size of a feature, while the number of outside commits is less predictable.
This observation supports the notion that the proportion of outside to inside commits of a feature and its size are negatively correlated.
While we determined the according linear correlation coefficients to be negative for all projects, they are rather high and their respective p-values too low to warrant significant evidence.
Still, most data points to the existance of a, ableit slight, negative correlation, which could be confirmed by investigating more projects. \\
During the evaluation of the first part of RQ2, we found that commits structurally interacting with features have an increased probability of influencing other features via outside dataflow.
To continue our investigation in this regard, we determined the proportion of outside commits of features that are part of structural CFIs as well.
The average proportions are 0.41, 0.59, 0.75 and 0.43 for the features of \textsc{xz}, \textsc{gzip}, \textsc{bzip2} and \textsc{lrzip}, respectively.
All features except \textsf{force} of \textsc{gzip} have proportions higher than 0.2 and the majority of values range close to the average of a project.
We can therefore confirm that a significant proportion of the data that flows into a feature probably originates from or is modified in other features.
This also means that features are less likely to be influenced by commits not changing any feature-code than our determined numbers of outside commits would suggest. \\
The regression plots of \textsc{gzip} show that their strong linear correlations are mainly caused by two clusters of features.
One cluster consists of large features with many inside and comparatively few outside commits, the other of features with a small size and significantly more outside than inside commits.
In the evaluation of RQ1, we have already explained that, due to feature-nesting, the sets of commits that structurally interact with each feature from the first cluster are almost identical.
This suggests that the inside commits of said cluster are almost identical as well and that this could also apply to the outside commits of the two clusters.
Indeed, we determined the majority of outside commits from the features of each cluster to be identical.
The respective percentages are $\geq$60\% for the cluster encompassing the large features and $\geq$70\% for the cluster of the smaller features excluding \textsc{recursive}.
Thus, the two clusters are not formed by coincidence, but because features of a similar size are largely affected by the same set of outside commits.
For the first cluster, this phenonemon can be explained by the fact that its features share large portions of their code, which results in them interacting with the same commits by design.
At the same time, this is not, or only partially, the case for the features of the second cluster.
For them, we note that 25 inside and 17 outside commits from features of the first cluster make up almost all of their identical outside commits.
Interestingly, this relation is especially apparent for the smallest feature of \textsc{gzip}, \textsf{no\_time}, which only encompasses 23 instructions and is fully located inside of the regions of the \textsf{force}-feature.
All of the 65 commits affecting \textsf{no\_time} through outside dataflow are inside or outside commits of \textsf{force}.
Thus, we find that there also exists a lot of dataflow between features of the two clusters, where the code of the \textsf{no\_time}-feature acts as a main intersection of said dataflow.
This is another product of \textsc{gzip}'s high degree of feature-nesting, where the features of the second cluster are often completely nested inside the larger features of the first cluster.
The data stemming from commits interacting with these large features, eventually passes through the smaller features within them, making said commits interact with the nested features through outside dataflow.
A high degree of feature-nesting can therefore not only lead to many features sharing the same structurally interacting commits, but also the commits affecting them through outside and inside dataflow.
% Abschließend ist festzuhalten, dass feature-nesting ebenfalls Auswirkungen auf die Interepretierbarkeit von den commits die feature über datenfluss beeinflussen, hat ->> Wieso?

\subsection*{\textbf{RQ3: Discussion of Author Interactions}}\label{sec:disc_author_interactions}

We find that our dataflow analysis was able to reveal many additional developers that did not directly participate in the development of a feature, but rather contributed commits whose data is later used inside features.
In most features of \textsc{gzip}, \textsc{bzip2} and \textsc{lrzip}, we detect a similar or higher number of unique dataflow authors compared to the number of structural authors.
While most unique dataflow authors are naturally found for features with more dataflow than structural authors, there also exist unique dataflow authors for features with more structural authors.
That is because, for the three mentioned projects, only slightly over 50\% of structural authors also interact with their respective features through outside dataflow.
This means that, while likely, developers of features do not always contribute commits affecting them through outside dataflow.
They also are not the ones generally introducing these commits, as the majority of features encompass more unique dataflow authors than structural authors. \\
The only statistically significant correlations relating the size of a feature to its number of interacting authors were produced by \textsc{gzip}.
Here, we are able to gather initial evidence that the number of authors implementing a feature strongly increases with its size.  
In contrast, we found the opposite to be true for the number of authors influencing it through outside dataflow. 
However, as only a single project produced statistically significant results, we consider the question to be completely open to further research. 
\textsc{xz} and \textsc{bzip2} are unsuitable for the computation of a significant correlation due to their small number of different authors interacting with features and \textsc{lrzip} due to its small number of features. \\
\textsc{gzip}'s strong positive correlation in regards to its structural authors matches its correlation between the size of a feature and its number of structurally interacting commits in RQ1.
This is also the case for \textsc{gzip}'s strong negative correlation regarding its dataflow authors and the correlation calculated for the number of outside commits in RQ2.
Overall, we notice that the number of authors interacting with a feature reflects the length of the set of commits from which we have extracted the authors.
In \textsc{gzip}, the two are positively correlated with a coefficient of 0.73 and a p-value of 0.004 for the number of structural commits and authors and a coefficient of 0.54 and a p-value of 0.058 for the number of outside dataflow commits and authors.
Thus, the number of commits interacting with a feature can hint at the number of authors that interact with it.
One reason for these strong positive correlations is the high degree of feature-nesting, which caused features of the two discussed clusters of \textsc{gzip} to have largely identical sets of interacting commits.
As the set of authors for a particular feature is extracted from the commits interacting with it, it is unsurprising that the authors interacting with features of each cluster are also largely identical.
This applies to 85 to 100\%, depending on the respective feature, of the structural authors of the four largest features. 
Similarly, the percentage of identical dataflow authors is 67-80\% for the first cluster and 75-100\% for the cluster of smaller features excluding \textsf{recursive}.
Thus, a high degree of feature-nesting can lead to many features having the same set of interacting authors.
This makes it difficult to trace whether, a certain set of authors has really implemented a certain set of features or whether, for example, only one author from the set has implemented one feature from the set. \\
In \textsc{gzip}, a total of 14 authors interact with features in any way, from which 13 structurally interact with features and 12 affect features through outside dataflow.
Only one author that did not touch any feature-code, affects features through dataflow and only two authors structurally interacting with features do not affect other features through outside dataflow.
Here it is important to note that \textsc{gzip} and its used library \textsc{gnulib} have 14 unique contributors, which means that all of its developers interact with features either structurally or through dataflow or both.
This is not surprising given \textsc{gzip}'s high extent of feature-code making up the majority of its instructions.
That the extent of feature-code plays a role in what share of a project's authors interact with features can also be seen for \textsc{bzip2} and \textsc{lrzip} with an extent of 8.2\% and 1\% respectively.
While there are 19 and 27 contributors in \textsc{bzip2}'s and \textsc{lrzip}'s git-repository respectivly, there only exist 4 unique structural authors of features in both projects.
The fact that \textsc{bzip2} only encompasses 3 unique outside dataflow authors, while \textsc{lrzip} encompasses 8, likely has to do with \textsc{lrzip} having many more active commits and overall contributors.

\section{Threats to Validity}\label{sec:threats}

There are some potential threats to the internal validity of our gathered data, which stem from our implementation in VaRA. \\
From the definition of \nameref{def:feature_regions}, it follows that we implement feature regions in such a way that any instruction, whose execution depends on a configuration variable, is part of a feature region.
However, not every such instruction also implements the functionality of a feature, as can be seen in listing~\ref{lst:feature_region_overapproximation}.
This means that the regions of a feature can overapproximate the instructions responsible for its functionality.
If the instructions, the CFI of a feature occurs in, are such overapproximated instructions, it follows that said CFI is overapproximated as well.
Since feature regions are used for computing both structural and dataflow-based CFIs, our generated reports probably contain overapproximated CFIs.
Furthermore, \citet{sattler2023seal} explains that our deployed VaRA taint analysis does not necessarily detect all dataflows occuring in a program.
This results in taints being underapproximated, meaning that some instructions are not tainted when they correctly should be.
Thus, some dataflow interactions could be missed by our deployed commit-feature interaction analysis. 
The detection and tracking of configuration variables stored in structs is currently being implemented in VaRA and was not usable for this work.  
Specifically, the configuration variables of some features in \textsc{xz} and \textsc{lrzip} cannot be detected, meaning that no CFIs can be collected for them. 
Logically, this affects the overall and project-specific results of our RQs.
For example, the percentage of commits with only one feature-related concerns are likely lowered by considering less features. \\
% Especially in the case of \textsc{lrzip}, a higher number of features could provide more statistically significant results by increasing the number of datapoints in our linear correlations \\
Concerning the external validity of our findings, most dangers stem from the selection of projects we investigate.
Our pool of investigated projects is limited and it is likely that the way commits and features are used in them is different to other projects to some extent.
In previous chapters, we mentioned that the chosen projects are rather small and from a compression domain.
This could mean that our findings might not be applicable for projects of larger sizes or of different domains.
As we already factor in the size of a feature in the analysis of our data, we are able to mitigate some doubts about the applicability of our results onto larger projects. 
Apart from the general selection of our exmained projects, \textsc{bzip2} in particular is a project whose results must be treated with caution.  
In the \nameref{sec:discussion} of RQ1, we already explained that the overall project and its features were largely already implemented when the actual development in the git-repository started.
This is reflected in a total of only 37 active commits and a maximum of 7 interacting commits for a feature. 
Accordingly, the results of \textsc{bzip2} differ significantly from those of the other projects. 
It was important for us to describe this fact accordingly so that we can justifiably attribute less weight to the results of \textsc{bzip2}.  
It is worth mentioning that the git-development of the other three projects also began with a pre-existing version.  
However, their features were not or only partially implemented at that point.
Their entire code has also changed significantly since then, as can be deducted from the high number of active commits.
We do therefore not consider this circumstance as a major threat to the validity of the results for \textsc{xz}, \textsc{gzip} and \textsc{lrzip}. \\

\begin{lstlisting}[language=python, caption={Feature Region Overapproximation. The function of \texttt{FeatureEncryption} is to send the message encrypted. According to our definition of feature regions all instructions stemming from the shown lines of code belong to a region of \texttt{FeatureEncryption}, as their execution depends on the configuration variable of \texttt{FeatureEncryption}. However only instructions stemming from the lines 1-2 implement the actual functionality of the feature. Thus our analysis overapproximates the lines 3-4 to also belong to the feature.}, label={lst:feature_region_overapproximation}]
1. if FeatureEncryption:                              %\vartriangleright% %FeatureEncryption%
2.    sendEncryptedMessage(message)                   %\vartriangleright% %FeatureEncryption%
3. else:                                              %\vartriangleright% %FeatureEncryption%
4.    sendMessage(message)                            %\vartriangleright% %FeatureEncryption%
\end{lstlisting}
\label{lst:feature_region_overapproximation

