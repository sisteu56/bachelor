%************************************************
\chapter{Evaluation}\label{ch:evaluation}
%************************************************

This chapter evaluates the thesis core claims.
- for this, we investigate four projects

\section{Results}\label{sec:results}

\subsection*{\textbf{RQ1: Evaluation of Structural CFIs}}\label{sec:eval_struc_cfis}

\subsubsection*{Patterns around Feature Development}\label{sec:eval_feature_development}

Aside from the number of commits used during its development, we can also extract the size of a feature from its structural CFIs.
These are the main properties we consider when investigating patterns around feature development.
Additionally, we factor in the nesting degree of structural CFIs during our analysis allowing us to extract more information from both properties.
Figure \ref{fig:feature_sfbr_plot} illustrates our results in three different plots for each project.
Each row displays the results for one project with the name of the respective project being shown on the far left. \\
In the first column, we show the number of structurally interacting commits for each feature in a bar-plot.
We differentiate between interactions with a nesting degree of 1, which are colored in blue, and ones with a higher nesting degree colored in orange.
The orange bar represents commits for which we are less certain of their specific purpose, where they could have been used to implement its respective feature and others at the same time. 
Overall, we notice a large spread of structurally interacting commits between features across all projects.
Most interactions occur at a nesting degree of 1, especially for features with many interacting commits in comparison to other features of their respective project.
Interestingly, for features with few interacting commits, a large proportion of their interactions have a higher nesting degree than 1.
We also notice that the range of interacting commits varies greatly between the different projects.
In \textsc{bzip2}, the highest number of interacting commits is 7 for the \textsc{verbosity} feature, but at over 60 for the \textsc{force,no\_name} feature in \textsc{gzip}.
The lowest number of interacting commits is 1 for all projects, except lrzip with 6, indicating that some features need very little work to be implemented.
The projects \textsc{xz} and \textsc{bzip2} bear some similarities, such as the \textsc{Verbosity}-feature having the highest number of interacting commits.
Both \textsc{xz} and \textsc{bzip2} also share a \textsc{ForceOverwrite}-feature, which has an average number of interacting commits in both projects, with a large portion of interactions having a nesting degree higher than one. \\
In the second column, we display the different feature sizes for each project in a bar-plot.
The y-ticks represent the number of instructions implementing a feature, i.e. the number of instructions stemming from its feature regions.
The blue bar represents the definite size of a feature, where we only count instructions that are exclusively part of its respective feature.
The orange bar adds the potential size of a feature on top, for which we also count instructions that are part of multiple feature regions at once.
For these instructions there is less certainty whether they really implemented functionality of that specific feature.
It also allows us to see how common feature nesting is inside a project and which features are nested inside other features.
Upon examining the shown feature sizes, we notice a pattern similar to that of the first column.
Here, larger features mostly consist of definite feature size, whereas smaller features have a high proportion of potential feature size.
This is not the case for \textsc{bzip2}, where all features have a high amount of potentially implementing instructions.
Surprisingly, this is also true for features that only have commits structurally interacting with them at nesting degree one.
The largest combined feature size we encounter is above 5000 for the feature \textsc{force,no\_name}, which is also the feature with most interacting commits.
For the projects \textsc{xz}, \textsc{bzip2} and \textsc{lrzip} the average feature size ranges between 200 and 400, while \textsc{gzip} has an average around 1.200.
This is the due to the very large feature we mentioned, which encompasses 4000 more instructions than the second largest feature. \\
In the third column of figure \ref{fig:feature_sfbr_plot}, the datasets used in the two previously dicussed plots are related inside two regression plots.
Broadly speaking, we plot the size of a feature against the number of commits that structurally interact with it.
For the regression plot colored in blue, we only consider the definite size of a feature and commits structurally interacting with them at a nesting degree of one.
The values of the displayed scatter points are identical with the values of the blue bars in the two previous columns. 
In the second regression plot, we consider all structurally interacting commits and all instructions regardless of any nesting degree.
As we now factor in the combined values of the blue and orange bars of the previous columns, the color of the plot is a combination of the two.
The respective correlation coefficents and p-values are shown at the bottom-right of each graph.
They are calculated with the pearson correlation coefficent, which measures the linear relationship between two datasets.
Across all projects and both linear regressions, we note a clear positive correlation between the size of a feature and the number of structurally interacting commits.
For both linear regressions, the correlation coefficients are quite high, ranging from ca. 0.9 for \textsc{xz} to ca. 0.98 for \textsc{lrzip}.
\textsc{bzip2} makes for the only exception here, where the correlation coefficent drops from 0.89 for the broader to 0.59 for the more specific consideration.
The p-values are very low for all pearson correlations ranging from close to zero for \textsc{xz} to 0.16 for \textsc{lrzip} indicating a high probality that both datasets are not un-correlated.
Combined with the high correlation coefficients, this gives us high confidence that size of a feature and the number of commits used during its development are strongly positively correlated.

\clearpage

\begin{figure}[htbp]
  \centering
  \includesvg[height=19cm]{gfx/results/RQ1/feature-sfbr-plot.svg}
  \caption{Feature Structural CFIs Plot}
  \label{fig:feature_sfbr_plot}
\end{figure}

\subsubsection*{Usage of Commits in Feature Development}\label{sec:eval_commit_usage}

\subsection*{\textbf{RQ2: Evaluation of Dataflow-based CFIs}}\label{sec:eval_df_cfis}

\subsubsection*{Proportion and Dependencies of Commits Affecting Features through Dataflow}\label{sec:eval_commit_dfbr}

The initial step in evaluating what fraction of commits affect features through dataflow, is determing the number of active commits.
Active commits are represented by at least one commit region inside a program, meaning that they fullfil the minimum requirement to be part of a dataflow-based CFI.
The respective values are shown in the first column of table \ref{tab:commit_dfbr_table} for each project.
We determined \textsc{xz} to have the highest number of active commits at 1039, while \textsc{bzip2} only has a tiny fraction of that at 37.
In the first plot of figure \ref{fig:commit_dfbr_plot}, the bars colored in red show what percenateg of commits interact with features through dataflow.
We notice that the respective percentages are vastly different from project to project.
The majority of \textsc{gzip}'s active commits are part of dataflow-based CFIs at 53.6\%.
This percentage gets halved for \textsc{bzip2} with about 27\%, with another large drop to 11.3\% for \textsc{xz}.
This means that roughly every second commit in \textsc{gzip}, every fourth commit in \textsc{bzip2} and every ninth commit in \textsc{xz} affects features through dataflow.
The bars colored in grey display the fraction of active commits structurally interacting with features.
In addition to the obvious fact of how often commits are used to implement features, this also gives us an estimation on the extend of feature-code in a project.
The more commits are part of structural CFIs, the more of the code contributed by them and therefore the overall code of a project will be part of feature regions.
Logically, the accuracy of this estimation depends on many factors, such as the extend to which commits contributing code to features, also contribute code to other parts of a program.
Still, given the large disparity in the percentage of commits with structural interactions, we can be relatively certain that \textsc{xz} has the lowest proportion and \textsc{gzip} the largest proportion of feature-code with \textsc{bzip2} somewhere inbetween.
Comparing the two bar-types, we notice that the percentage of commits with structural interactions is lower than the percentage of commits with dataflow interactions for each project.
This phenomenon can be explained considering the values presented in the second column of table \ref{tab:commit_dfbr_table}.
There, we show the probablity for a commit to be part on any dataflow-based CFI, given that said commit is part of any structural CFI.
We see that the probabilities are roughly the same for each project at around 90\%.
This means that by only taking into account commits part of structural CFIs, we already encounter a lot of commits that affect features through dataflow.
If we then consider the entire set of active commits, the number of commits with dataflow interactions will likely exceed the number of commits with structural interactions.
Across all projects, the probability of a commit interacting with features through dataflow, given that it is part of structural CFIs, is much higher than the same probability for any active commit.
This is a clear indication that our assumptions in section \ref{sec:combination_cfis}, that structural CFIs heavily coincide with dataflow-based CFIs, are correct.
Since our intent is to especially focus on commits whose interactions with features can only be discovered by employing our dataflow analysis, we now aim to quantify these commits in the investigated projects.
For this, we examine the relative difference between the percentages of commits with structural and commits with dataflow interactions.
This difference roughly determines what share of commits are part of the discussed more interesting dataflow interactions.
The difference is the smallest for \textsc{bzip2} at 10\% and slightly higher than that for \textsc{gzip} at 15\%.
\textsc{xz} has by far the biggest relative difference at almost 49\%, which means that our dataflow analysis reveals many additional interactions here.
We explore the topic of different dataflow interaction types more thoroughly in our second plot, which focuses on the \textsf{origin} of dataflow. \\
There, we have a closer look at the commits affecting features through dataflow and where their dataflow stems from.
Naturally, dataflow occuring inside the regions of a feature is more intentional and therefore less interesting, than dataflow originating outside its regions.
In section \ref{sec:combination_cfis}, we explain, that the dataflow of commits and features not structurally interacting with each other, must be outside dataflow.
Even though the regions of a commit and regions of a feature must partially overlap in order for them to structurally interact, such a commit can still have regions not part of any feature regions.
Therefore, we cannot be sure whether the dataflow of these interactions originates from outside or inside the regions of a feature.
Due to the inherent properties of structural interactions, i.e. heavily coinciding with dataflow interactions, we assume their dataflow to originate inside the regions of a feature.
In the second plot of figure \ref{fig:commit_dfbr_plot}, we separate the commits with dataflow interactions into three categories based on dataflow origin.
The first category, shown as the bar colored in blue, represents commits that only interact with features through outside dataflow.
They make up the majority of commits for \textsc{xz} at 56.4\% and only 20\% of commits for both \textsc{bzip2} and \textsc{gzip}.
The second category represents commits that interact with features through outside and inside dataflow.
Logically, these commits affect at least two features through dataflow and only structurally interact with a subset of them.
Surprsingly, they form the majority of commits for \textsc{bzip2} and \textsc{gzip} at 60 and 51\% respectively and around 20\% of commits for \textsc{xz}.
The proportion of commits that only interact with features through inside dataflow varies less between the projects.
\textsc{gzip} has the highest percentage at 28.8\%, \textsc{bzip2} the lowest at 20\% with \textsc{xz} inbetween the two at 23.9\%. \\
We admit that the high proportions of commits with both inside and outside dataflow are rather unexpected.
They hinder a clear separation of commits with dataflow interactions into less and more interesting commits, e.g. commits whose dataflow interactions are more and less obvious.
We notice that commits interacting with features through inside dataflow, have a high chance to also interact with other features through outside dataflow.
By definition, all commits interacting with features through inside dataflow also structurally intercact with them.
From previous explanations, we also know these commits are almost identical to the entire set of commits with structural interactions.
It follows that commits part of structural CFIs must also have a high, albeit slightly lower, chance to affect features through outside dataflow.
We compare this probability to the probability of any active commit to interact with features through outside df in table \ref{tab:commit_dfbr_rel_table}.
The strongest contrast occurs for \textsc{xz}, where 38\% of commits structurally interacting with features, interact with other features through outside df, making them 4.5 times more likely to do so compared to any active commit of \textsc{xz} at 8.5\%.
The contrast is slightly weaker for \textsc{bzip2} with a 3.5 times higher likelihood and according percentages of 66.7 and 21.6\% respectively. 
\text{gzip} has the weakest contrast at 2.2 and respective probabilites of 60.2 and 26.8\%. 

\begin{figure}[htbp]
  \centering
  \includesvg[height=8cm]{gfx/results/RQ2/proportional_commit_dfbr_plot.svg}
  \caption{Proportional Dataflow-Plot for Commits}
  \label{fig:commit_dfbr_plot}
\end{figure}

\begin{table}[t]
\caption{Additional Information for Dataflow Analysis of Commits}
\label{tab:commit_dfbr_table}
\begin{tabular}{llrr}
 Projects & Number of Active Commits & Probability for Commit to be part of dataflow CFI \\
 	  & 			     & given that it is part of structural CFI \\
xz & 1039 & 0.883 \\
gzip & 194 & 0.933 \\
bzip2 & 37 & 0.889 \\
\end{tabular}
\end{table}

\begin{table}[t]
\caption{Relating Inside Dataflow to Outside Dataflow}
\label{tab:commit_dfbr_rel_table}
\begin{tabular}{llrr}
 Projects & Probability for Commits Part of Structural CFIs & Same Probability Given 	\\
 	  & to Affect other Features Through Outside DF     & Any Active Commit		\\
xz & 0.38 & 0.085 \\
gzip & 0.602 & 0.268 \\
bzip2 & 0.667 & 0.216 \\
\end{tabular}
\end{table}

\subsection*{\textbf{RQ3: Evaluation of Author Interactions}}\label{sec:eval_author_interactions}

\section{Discussion}\label{sec:discussion}

In this section, discuss your results.

\section{Threats to Validity}\label{sec:threats}

In this section, discuss the threats to internal and external validity.
