%************************************************
\chapter{Introduction}\label{ch:introduction}
%************************************************

Features play an important part in modern programming, which shows in program paradigms such as feature-oriented software development (FOSD),
where they are used to implement specific functionalities, can be activated or deactivated and thus add configurability to software systems.
As commits are used to contribute new source-code to a repository, it follows that commits can structurally interact with features by introducing source-code implementing their functionality. 
Furthermore, seemingly unrelated code, that was changed or added by commits, might influence soure-code of a feature through dataflow.
We aim to give initial insights into these topics by investigating \nameref{ch:example_chapter} (CFIs).

Within a program, there exist many different abstract entities, such as commits and features, each serving different responsibilites.
For a better understanding of them, it is advantageous to know whether they interact with or among each other.
In the context of CFIs, bugs occuring in a feature could be linked to the latest commits affecting data of said feature, and consequently the authors responsible for these bugs.
Especially for dataflow spanning over multiple files and many lines of code, it might be difficult to determine the respective interactions by studying the program yourself.
To enable an automatic detection of these interactions and allow for a complete overview of them inside a software project, Sattler et al.\cite{sattler2023thesis} created the interaction analysis tool VaRA\cite{VaRA2023}.

In a prior study, \citet{sattler2023seal} used VaRA to examine interactions between commits showing that research on this topic can be applied to improve many aspects of software development.
For example, their research allowed for a deeper understanding of interactions between developers by linking commit interactions to their respective authors\cite{sattler2023seal}. 
Similarly to commit interactions, internal feature interactions are dependencies between features at source-code level. 
While they have been the topic of some studies\cite{kolesnikov2017relation}, the main intention was to utilize them as a predictor for external, e.g. performance interactions\cite{siegmund2012predicting} of features.

We pick up and extend on previous research by being the first to investigate interactions between commits and features.
For this, we introduce the concept of structural and dataflow-based CFIs.
Structural CFIs occur inside a program when there exists overlap between source-code changed by a commit and source-code constituting a feature.
This overlap implies that the commit changed or implemented functionality of the feature, meaning that structural CFIs can produce data on feature development.
We speak of dataflow-based CFIs when there exists dataflow from the code representations of commits to that of features.
Thus, commits of these interactions introduced changes in the program affecting data, which is later used inside a feature.
CFIs based on dataflow examine the program on a deeper layer, allowing us to detect additional interactions missed by structural CFIs.
By combining dataflow-based and structural interactions, we can specify whether dataflow stems from outside or from inside the code constituting a feature, thus producing more informative data.

In a second analysis step, we follow the discussed approach of Sattler et al. by enriching CFIs with high-level repository information.
In the same way in which we examine the influence of commits on features, we can now examine how the authors contributing these commits interact with features.
As with commits, we say that developers can affect features structurally or through dataflow.
There are some important differences though, as a developer can be the author of multiple commits, which can in turn affect features in different ways.
Thus, developers implementing features might also be the ones contributing changes located outside of feature-code influencing them through dataflow.

\section{Goal of this Thesis}

The primary focus of this thesis is to gain an overview of how commits interact with features in software projects. 
Our goal is to lay basic groundwork regarding this subject, while leaving more detailed questions to future research. 
As previously mentioned, we investigate two types of commit-feature interactions, namely structural and dataflow-based CFIs. 
Using structural CFIs, we aim to reveal insights about the development process of features and usage of commits therein. 
We also investigate to what extend our dataflow analysis can reveal interactions that cannot be discovered with a purely syntactical analysis.
By examining the interplay between the two types of CFIs, we hope to establish dependencies that will be considered in future studies.
Finally, we provide first insights on interactions of authors with features to incentivize that this subject could be worthy of further investigation.

\section{Overview}

The \hyperref[ch:background]{Background} chapter serves as an introduction to the concepts of code regions and interaction analysis, which are necessary to properly define structural and dataflow-based CFIs.
Their definition then takes place in the \hyperref[ch:example_chapter]{Commit-Feature Interactions} chapter, where we thouroughly dicuss their meaning in software projects as well as our \nameref{ch:implementation} of their detection in VaRA.
We also define properties surrounding features and commits, that are used in the following chapters, like the size of a feature~(\ref{sec:feature_size}) and the feature-related concerns of a commit~(\ref{sec:commit_concerns}).
The research questions of this paper are established in the \hyperref[ch:methodology]{Methodology} chapter, in which we also explain their investigation.
The \hyperref[ch:evaluation]{Evaluation} includes a detailed description of our results for the three posed research questions, followed by a separate discussion of the findings for each RQ. % containing background information and possible reasons for them.
In the last section of \autoref{ch:evaluation}, we address potential internal and external \hyperref[sec:threats]{Threats to the Validtiy} of this work.
In \hyperref[ch:related_work]{Related Work}, we summarize previous studies investigating interactions inbetween commits and inbetween features.
We also cover topics such as the detection methods of feature-code and previous research on feature development. % to be done 
To finish the thesis, we summarize our results as well as their discussion and formulate thoughts adressing future research in our \hyperref[ch:conclusion]{Concluding Remarks}.
