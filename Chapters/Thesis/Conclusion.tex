%************************************************
\chapter{Concluding Remarks}\label{ch:conclusion}
%************************************************

\section{Conclusion}\label{sec:conclusion}

\section{Future Work}\label{sec:futurework}

- in this work, some expressions can be ambigious due to limits in our analysis and a high degree of feature-nesting \\
- by improving our analysis and finding ways to decipher nested features, future research could prevent these ambiguities \\
- we say that a commit affects a feature through inside dataflow, meaning that the dataflow occurs inside the regions of the feature, when it also structurally interacts with the feature \\
- as the commit could also be partially located outside the feature, we could be dealing with an outside dataflow origin as well \\
- one way to solve this issue could be to track whether the commit taint, belonging to data being used within a feature, originated inside the regions of said feature \\
- for this, we need to assign such a taint the feature regions of the instruction it stems from \\
- if 

- in regards to feature-nesting, it would be extremely helpful to know which set of features is implemented by an instruction part of multiple feature regions \\
- here, a qualitive analysis of nested features is necessary to detect whether there exist to patterns that could help us determine which nested feature is being implemented \\
- gaining information form the developers of these features could provide us with important insights \\
- for example, we could say that features, whose regions are the most nested in a set of consecutive instructions, are currently being implemented \\
-

- regarding dataflow interactions, we have only investigated dataflow in one direction, namely from commits to features \\
- combining this with an investigation of dataflow from features to commits could tell us which commits connect data from different features \\
- we suppose that commits that are affected by and affect several features through outside dataflow are central to the interplay of the involved features \\
- as we have seen evidence that dataflow between features is quite common, there should exist many such commits \\

- more projects and of different domains => solidy or revise our results \\
- at least 10 features such that correlations are more likely to be statistically significant \\
- some RQs such as the correlation between the size of a feature and its proportion of outside to inside commits still left to be answered => more datapoints needed \\


