%************************************************
\chapter{Concluding Remarks}\label{ch:conclusion}
%************************************************

% \section{Conclusion}\label{sec:conclusion}

% \section{Future Work}\label{sec:futurework}

- in this thesis, we achieved an initial overview of CFIs in software projects, specifically structural and dataflow-based CFIs, their interplay as well as other properties influencing them \\
- we were able to answer or at least gain some interesting insights into numerous research topics we set out to investigate \\

- in RQ1, we found that the number of commits used during their development varies strongly between the features of a project \\
- here, the size of a feature is an accurate predictor for the number of commits as the two metrics are strongly positively correlated \\
- regarding the usage of commits during feature development, we support the notion that they usually change a single feature as the majority of them only have one feature-related concern \cite{} \\

- similarly to the number of implementing commits, the number of commits affecting a feature through outside dataflow also encompasses a wide range among the features of a project \\
- admittedly, our analysis potentially underestimates the number of commits whose dataflow stems from the outside, although this could be fixed in future research \\
- we say that a commit affects a feature through inside dataflow, meaning that the dataflow occurs inside the regions of the feature, when it also structurally interacts with the feature \\
- as the commit could also be partially located outside the feature, we could be dealing with an outside dataflow origin instead \\
- one way to solve this issue could be to track whether the commit taint \ref{}, belonging to data being used within a feature, originated inside the regions of said feature \\
- we found conflicting evidence regarding the linear correlation between the size of a feature and its number of outside commits interacting with it through dataflow \\
- all of our four examined projects produce unique results ranging from positive to no and even strong negative correlations \\
- further research is necessary to determine whether there exists an accurate predictor for a feature's number of outside commits where considering properties of a feature besides its size might be required \\
% which properties
- likewise, our results regarding the proposed decrease in the proportion of outside to inside commits in a feature with an increase in its size are not conclusive \\
- while most evidence points to existence of such a relation, more projects must be examined to gather statistically significant data \\

- the fraction of commits part of dataflow interactions are not unanimous among the projects and can be unexpectedly high \\
- they range from every ninth commit in \textsc{xz} to more than every second commit in \textsc{gzip} \\
- furthermore, we found that their fraction in a project is strongly influenced by the percentage of commits that structurally interact with features \\
- specifically, we proposed that said percentage acts as a lower bound for them since we could confirm the notion that structural CFIs heavily coincide with dataflow-based CFIs \ref{} \\
- still, our dataflow analysis revealed previously hidden interactions for the majority of commits interacting with features in all projects \\
- nonetheless, future research should factor in the discussed dependency between structural and dataflow interactions to properly assess the amount of new information a dataflow analysis can contribute \\
% to differentiate between less and more interesting CFIs...
- by considering structural CFIs in RQ2, we were also able to determine that commits used to implement features have an increased likelihood to affect other features via outside dataflow \\
% => the same is true for authors













\iffalse 
- in this work, some expressions can be ambigious due to limits in our analysis and a high degree of feature-nesting \\
- by improving our analysis and finding ways to decipher nested features, future research could prevent these ambiguities \\
- we say that a commit affects a feature through inside dataflow, meaning that the dataflow occurs inside the regions of the feature, when it also structurally interacts with the feature \\
- as the commit could also be partially located outside the feature, we could be dealing with an outside dataflow origin as well \\
- one way to solve this issue could be to track whether the commit taint, belonging to data being used within a feature, originated inside the regions of said feature \\
- for this, we need to assign such a taint the feature regions of the instruction it stems from \\
- if 

- in regards to feature-nesting, it would be extremely helpful to know which set of features is implemented by an instruction part of multiple feature regions \\
- here, a qualitive analysis of nested features is necessary to detect whether there exist to patterns that could help us determine which nested feature is being implemented \\
- gaining information form the developers of these features could provide us with important insights \\
- for example, we could say that features, whose regions are the most nested in a set of consecutive instructions, are currently being implemented \\
-

- regarding dataflow interactions, we have only investigated dataflow in one direction, namely from commits to features \\
- combining this with an investigation of dataflow from features to commits could tell us which commits connect data from different features \\
- we suppose that commits that are affected by and affect several features through outside dataflow are central to the interplay of the involved features \\
- as we have seen evidence that dataflow between features is quite common, there should exist many such commits \\

- more projects and of different domains => solidy or revise our results \\
- at least 10 features such that correlations are more likely to be statistically significant \\
- some RQs such as the correlation between the size of a feature and its proportion of outside to inside commits still left to be answered => more datapoints needed \\
\fi

