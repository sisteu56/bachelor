%************************************************
\chapter{Background}\label{ch:background}
%************************************************

In this section, we summarize previous research on the topic of code regions and interaction analysis.
Thereby, we give defintions of terms and dicuss concepts that are fundamental to our work.
In the next chapter, we use the introduced definitions and concepts to explain and investigate structural as well as dataflow-based commit-feature interactions.

\section{Code Regions}\label{ch:code_regions}
Software programs typically get translated into machine code by a compiler.
Some compilers translate the source-code of a program into an intermediate representation (IR) before that to simplify the implementation of code optimizations.
IR plays an important role in our work, as the static program analysis we use is conducted on top of IR. \\
Code regions can be used to connect abstract entities of a software project, such as commits or features, to IR~\cite{sattler2023seal}.
They are comprised of IR instructions that are consecutive in their control-flow.
For this, each code region carries some kind of tag, detailing its meaning.
It should be noted that there can exist several code regions with the same tag scattered across the program.
We dicuss two kinds of tags, namely commit and feature tags, allowing us to define two separate code regions. 

\paragraph{Commit Regions}\label{sec:commit_regions}

Commits are used within a version control system to introduce the latest source code changes in its respective repository.
Inside a repository revision, a commit encompasses all source code lines that were added or last changed by it. 

\newtheorem{definition}{Definition}
\begin{definition}\label{def:commit_regions}
	\emph{The set of all consecutive instructions, that stem from source code lines belonging to a commit, is called a} commit region.
\end{definition}

As the source code lines changed by a commit are not necessarily contigious,
a commit can have many commit regions inside a program. 
To properly address the entire set of these commit regions within a program, we introduce the term regions of a commit.
We say that the \emph{regions of a commit} are all commit regions that stem from source-code lines of the commit. 

\paragraph{Feature Regions}\label{sec:feature_regions}

In general, features are program parts implementing specific functionality.
In this work, we focus on features that are modelled with the help of runtime configuration variables that specify whether the functionality of a feature should be active or not.
Inside a program, configuration variables decide whether instructions, performing a feature's intended functionality, get executed. 
Detecting these control-flow dependencies is achieved by deploying extended static taint analyses, such as Lotrack~\cite{lillack2014tracking}.

\begin{definition}\label{def:feature_regions}
	\emph{The set of all consecutive instructions, whose execution depends on a configuration variable belonging to a feature, is called a} feature region. 
\end{definition}

Similarly to commits, features can have many feature regions inside a program
as the instructions implementing their functionalities can be scattered across the program.
Here, we also say that the \emph{regions of a feature} are all feature regions that stem from a feature's configuration variable. 

\section{Interaction Analysis}\label{sec:interaction_analysis}

In the previous chapter, we have already shown how different abstract entities of a software project, such as features and commits, can be assigned concrete representations within a program.
We can use interaction analyses to infer interactions between these entities by computing interactions between their concrete representations (code regions).
This can improve our understanding of them and give us insights on how these entities are used inside software projects. 
An important component of computing interactions between code regions is the concept of interaction relations between them as introduced by \citet{sattler2023seal}.
As we investigate structural and dataflow-based interactions, we make use of structural interaction ($\circledcirc$) and dataflow interaction ($\rightsquigarrow$) relations in this work. 

The structural interaction relation between two code regions $r_1$ and $r_2$ is defined as follows:

\begin{definition}\label{def:structural_relation}
	\emph{The structural interaction relation $r_1 \circledcirc r_2$ evaluates to true if at least one instruction that is part of $r_1$
	is also part of $r_2$.}
\end{definition}

The dataflow interaction relation between two code regions $r_1$ and $r_2$ is defined as:

\begin{definition}\label{def:dataflow_relation}
	\emph{The dataflow interaction relation $r_1 \rightsquigarrow r_2$ evaluates to true if the data produced by at least one instruction i that is part of $r_1$
	flows as input to an instruction $i'$ that is part of $r_2$.}
\end{definition}

In interaction analyses, the computation of interactions between different entities is based upon abstract interaction relations, like the ones we have defined above.
Essential to the research conducted in this paper is the interaction analysis created by Sattler~\cite{sattler2023thesis}.
Their interaction analysis tool VaRA is implemented on top of LLVM and PhASAR.
In their novel approach SEAL~\cite{sattler2023seal}, VaRA is used to determine dataflow interactions between commits.
This is accomplished by computing dataflow interactions between the respective regions of commits according to the \hyperref[def:dataflow_relation]{dataflow interaction relation}.
Their approach for this will be shortly discussed here, however we advice their paper for a more thorough explanation. 

The first step of their approach is to annonate code by mapping information about its regions to the compiler's intermediate representation.
This information is added to the LLVM-IR instructions during code generation.
In their paper, \citet{sattler2023seal} focus on commit regions, which contain information about the commit's hash and its respective repository.
The commit region of an instruction is determined by the commit that last changed the source-code line the instruction stems from.
During the IR's construction, the commit of each source-code line is determined by accessing repository meta-data, after which the instructions stemming from each line are annotated with the hash of the respective commit~\cite{sattler2023seal}.

The second analysis step involves the actual computation of the interactions.
For this, Sattler et al. implemented a special, inter-procedural taint analysis.
It is able to track data flows between the commit regions of a given target program.
For this, information about which commit region affected it, is mapped onto data and tracked along program flow.
This is done by propagating taints, specifically commit taints, through the program.
Taints are used to to give information about which code regions have affected an instruction through dataflow.
For example, an instruction is tainted by a commit if its taint stems from a commit region.
It follows, that two commits, with their respective commit regions $r_1$ and $r_2$, interact through dataflow, when an instruction is part of commit region $r_2$, while being tainted by $r_1$.
Consequently data produced by the commit region $r_1$ flows as input to an instruction within the commit region $r_2$, matching the \lowercase{\text{\hyperref[def:dataflow_relation]{dataflow interaction relation}}}.
