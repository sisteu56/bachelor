%************************************************
\chapter{Commit-Feature Interactions}\label{ch:example_chapter}
%************************************************

In this section, we define structural and dataflow-based commit-feature interactions as well as properties related to them.
Furthermore, their meaning and relationship inside a software project is explained here. 
In the \nameref{ch:background} chapter, we discussed what purpose commits and features serve in a software project.
Commits are used to add new changes, whereas features are cohesive entities in a program implementing a specific functionality.
In this work, structural interactions are used to investigate how commits implement features and their functionality.
In addition, dataflow interactions are examined to gain additional knowledge on how new changes to a program, in the form of commits, affect features. 

\section{Structural CFIs}\label{sec:structural_cfis}

In the background chapter, we dicussed the concept of \nameref{ch:code_regions}, especially commit and feature regions. 
Logically, we speak of structural interactions between features and commits when their respective regions structurally interact. 
This structural interaction between code regions occurs when at least one instruction is part of both regions.
This is the case when code regions structurally interact through the \nameref{def:structural_relation} ($\circledcirc$).

\begin{definition}\label{def:structural_cfi}
\emph{A commit C with its commit regions $r_{1C}$, $r_{2C}$,... and a feature F with its feature regions $r_{1F}$, $r_{2F}$,... structurally interact, if at least one commit region $r_{iC}$ and one feature region $r_{jF}$ structurally interact with each other, i.e. $r_{iC}$ $\circledcirc$ $r_{jF}$.}
\end{definition}

Structural CFIs carry an important meaning, namely that the commit of the interaction was used to implement or change functionality of the feature of said interaction.
This can be seen when looking at an instruction accounting for a structural interaction, as it is both part of a commit as well as a feature region.
From the definition of \nameref{def:commit_regions}, it follows that the instruction stems from a source-code line that was last changed by the region's respective commit. 
From the definition of \nameref{def:feature_regions}, we also know that the instruction implements functionality of the feature region's respective feature. 
Thus, the commit of a structural interaction was used to extend or change the code implementing the feature of that interaction.
Following this, we can say that the commits, a feature structurally interacts with, implement the entire functionality of the feature.
That is because each source-code line of a git-repository was introduced by a commit and can only belong to a single commit.
Thus every instruction, including those part of feature regions, is annotated by exactly one commit region. 

Knowing the commits used to implement a feature allows us to determine the authors that developed it.
This is made possible by simply linking the commits, a feature structurally interacts with, to their respective authors.
By determining the authors of a feature, we can achieve a deeper insight into its development than solely focusing on the commits that implemented it. 

\section{Dataflow-based CFIs}\label{sec:dataflow_cfis}

Determining which commits affect a feature through dataflow can reveal additional interactions between commits and features that cannot be discovered with a structural analysis.
Especially dataflows that span over multiple files and many lines of code might be difficult for a programmer to be aware of.
Employing VaRA's dataflow analysis, that is dicussed in section~\ref{ch:implementation}, facilitates the detection of these dataflow interactions. \\
Commit interactions based on dataflow were explained in the \nameref{sec:interaction_analysis} section and can be considered as precursors to dataflow-based commit-feature interactions.
Similarly to commits interacting with other commits through dataflow, commits interact with features through dataflow, when there exists dataflow from a commit to a feature region.
This means that data allocared or changed within a commit region flows as input to an instruction located inside a feature region.
This pattern can also be matched to the \nameref{def:dataflow_relation} ($\rightsquigarrow$) when defining dataflow-based commit-feature interactions.

\begin{definition}\label{def:dataflow_cfi}
\emph{A commit C with its commit regions $r_{1C}$, $r_{2C}$,... and a feature F with its feature regions $r_{1F}$, $r_{2F}$,... interact through dataflow, if at least one commit region $r_{iC}$ interacts with a feature region $r_{jF}$ through dataflow, i.e. $r_{iC}$ $\rightsquigarrow$ $r_{jF}$.}
\end{definition}

\begin{lstlisting}[language=C++, caption={Illustration of structural and dataflow-based CFIs}, label=DescriptiveLabel]	
1. int calc(int val) {                             %\vartriangleright% %\texttt{d93df4a}%
2.    int ret = val + 5;                           %\vartriangleright% %\texttt{7edb283}%
3.    if (FeatureDouble) {                         %\vartriangleright% %\texttt{fc3a17d}%    %\vartriangleright% %FeatureDouble%
4.        ret = ret * 2;                           %\vartriangleright% %\texttt{fc3a17d}%    %\vartriangleright% %FeatureDouble%
5.    }                                            %\vartriangleright% %\texttt{fc3a17d}%    %\vartriangleright% %FeatureDouble%
6.    return ret;                                  %\vartriangleright% %\texttt{d93df4a}%   
7. }                                               %\vartriangleright% %\texttt{d93df4a}%   
\end{lstlisting}
\textsf{The above code snippet contains both structural as well as dataflow-based commit-feature interactions.
Commit \texttt{fc3a17d} implements the functionality of \texttt{FeatureDouble} for this function.
It follows that a structural commit-feature interaction can be found between them, as their respective commit and feature regions structurally interact.
Commit \texttt{7edb283} introduces the variable \texttt{ret} that is later used inside the feature region of \texttt{FeatureDouble}. 
This accounts for a commit-feature interaction through dataflow, as data that was produced within a commit region is used as input 
by an instruction belonging to a feature region of \texttt{FeatureDouble} later on in the program.}

\section{Combination of CFIs}\label{sec:combination_cfis}

When investigating dataflow-based CFIs, it is important to be aware of the fact that structural CFIs heavily coincide with them.
This means that whenever commits and features structurally interact, they are likely to interact through dataflow as well.
As our structural analysis has already discovered that these commits and features interact with each other, 
we are more interested in commit-feature interactions that can only be detected by a dataflow analysis.
That this relationship between structural and dataflow interactions exists becomes clear when looking at an instruction accounting for a structural CFI.
From definition~\ref{def:structural_cfi}, we know that the instruction belongs to a commit region of the interaction's respective commit.
It follows that data changed inside the instruction produces commit taints for instructions that use the data as input. 
Now, if instructions that use the data as input are also part of a feature region of the interaction's respective feature, the commit and feature of the structural interaction will also interact through dataflow.
However, such dataflow is very likely to occur, as features are functional units, whose instructions build and depend upon each other. 
Knowing this we can differentiate between dataflow-based CFIs that occur within the regions of a feature and those where data flows from outside the regions of a feature into it.
From prior explanations, it follows that this differentiation can be accomplished by simply checking whether a commit that influences a feature through dataflow, also structurally interacts with it. 

\section{Feature Size}\label{sec:feature_size}

When examining commit-feature interactions in a project, it is helpful to have a measure that can estimate the size of a feature.
We can use such a measure to compare features with each other and, thus, put the number of their interactions into perspective.
Considering our implementation, it makes most sense to define the size of a feature as the number of instructions implementing its functionality inside a program.
As the instructions inside the regions of a feature implement its functionality, we can the define the size of a feature as follows:
\begin{definition} \label{def:feature_size}
\emph{The} size \emph{of a feature is the number of instructions that are part of its feature regions.}
\end{definition}
It's possible to calculate the defined size of a feature by calculating the number of instructions in which structural CFIs occur.
That is, because every instruction that is part of a feature region accounts for a structural commit-feature interaction, as every instruction is part of exactly one commit region as shown in the beginning of this section.
It follows, that we do not miss any instructions that are part of feature regions and do not count any such instruction more than once. 

\section{Nesting Degree of Structural CFIs}\label{sec:nesting_degree}

In contrast to commit regions, multiple regions of different features can appear in a single instruction.
Here, we say that the number of feature regions inside an instruction equals its nesting degree.
With increasing nesting degree of an instruction, it becomes less certain which purpose the instruction serves specifically.
It could implement functionality of only a single feature present in the instruction's feature regions or any combination of them.
For an instruction with a nesting degree of one, we can be most certain about its purpose, namely implementing the functionality of the one feature present in the instruction.
This observation also bears consequences in the way we look at and judge structural CFIs, as we can also define a nesting degree for them:
\begin{definition} \label{def:nesting_degree}
\emph{The} nesting degree \emph{of a structural CFI is the minimum nesting degree of all instructions the interaction occurs in.}
\end{definition}
The nesting degree of a structural CFI is a useful measure to estimate the likelihood for which a commit was used to specifically change or implement functionality of a feature.
We can be most certain for a structural CFI with a nesting degree of one and become less certain for interactions with higher nesting degrees. \\

\begin{lstlisting}[language=C++, caption={Use-Case of Structural CFI's nesting degree}, label=DescriptiveLabel]	
26. if (CheckZero) {                    %\vartriangleright% %\texttt{b9ea2f3}%    %\vartriangleright% %CheckZero%
27.    if (x == 0) {                    %\vartriangleright% %\texttt{b9ea2f3}%    %\vartriangleright% %CheckZero%
28.       throw Error();                %\vartriangleright% %\texttt{b9ea2f3}%    %\vartriangleright% %CheckZero%
29.    }                                %\vartriangleright% %\texttt{b9ea2f3}%    %\vartriangleright% %CheckZero%
30. }                                   %\vartriangleright% %\texttt{b9ea2f3}%    %\vartriangleright% %CheckZero%
.   ...   
.   ...
.   ...
73. if (Min && CheckZero) {             %\vartriangleright% %\texttt{e49c7a1}%    %\vartriangleright% %Min,CheckZero%
74.    x = min(x, y);                   %\vartriangleright% %\texttt{e49c7a1}%    %\vartriangleright% %Min,CheckZero%
75.    if (x == 0) {                    %\vartriangleright% %\texttt{b9ea2f3}%    %\vartriangleright% %Min,CheckZero%
76.       throw Error();                %\vartriangleright% %\texttt{b9ea2f3}%    %\vartriangleright% %Min,CheckZero%
77.    }                                %\vartriangleright% %\texttt{b9ea2f3}%    %\vartriangleright% %Min,CheckZero%
78. }                                   %\vartriangleright% %\texttt{e49c7a1}%    %\vartriangleright% %Min,CheckZero%
\end{lstlisting}
\textsf{The above code-snippet illustrates how considering the nesting degree of a structural CFI can add important context to its meaning. \\ 
We can see that the commit \texttt{b9ea2f3} exclusively implements functionality of the \texttt{CheckZero}-feature. 
Still, it also structurally interacts with the \texttt{Min}-feature in lines 75-77. 
The according structural CFI only having a nesting degree of 2, allows us to be less certain of its specifc purpose. 
At the same time, the structural interaction between \texttt{b9ea2f3} and \texttt{CheckZero} has a nesting degree of 1, confirming that the interaction's commit was used to specifically implement the \texttt{CheckZero}-feature.}


\section{Implementation}\label{ch:implementation}

The detection of structural as well as dataflow-based commit-feature interactions is implemented in VaRA \cite{VaRA2023}.
Additionaly to commit regions, VaRA maps information about its feature regions onto the compiler's IR during its construction.
Commit regions contain the hash and repository of their respective commits, whereas feature regions contain the name of the feature they originated from.
VaRA also gives us access to every llvm-IR instruction of a program and its attached information.
Thus, structural CFIs of a program can be collected by examining its compiled instructions.
According to definition~\ref{def:structural_cfi}, we can store a structural interaction between a commit and a feature, if an instruction is part of a respective commit and feature region.
As discussed in the previous section, the nesting degree of an instruction equals the number of feature regions it is part of.
When encountering another instruction a structural interaction occurs in, we update the interaction's nesting degree as the minimum of its current value and the nesting degree of the inspected instruction.
For each structural CFI we also save the number of instructions it occurs in. 
This is accomplished by incrementing its instruction counter if we happen to encounter a duplicate. 

With this, we are also able to calculate the size of a feature based on our collected structural CFIs and their respective instruction counters.
Following the explanations from section~\ref{sec:feature_size}, we can compute the size of a feature as the sum over the instruction counters of all found structural CFIs the feature is part of. 

In the \nameref{sec:interaction_analysis} section, we discussed the taint analysis deployed by VaRA.
There, VaRA computes information about which code regions have affected an instruction through dataflow.
Checking whether a taint stems from a commit region allows us to extract information about which commits have tainted an instruction.
Thus, dataflow-based commit-feature interactions can also be collected on instruction level.
According to definition~\ref{def:dataflow_cfi}, we can store a dataflow-based interaction between a commit and a feature, if an instruction has a respective commit taint while belonging to a respective feature region.
Consequently said instruction uses data, that was changed by a commit region earlier in the program, as its input. 

jetzt auch VaRA-TS erwähnen ??
\iffalse
For our research we examine numerous software projects to get a wide range of reference data, as commit-feature interactions could potentially vary greatly between different code spaces.
Accordingly, the VaRA-Tool-Suite was extended making it possible to generate a report comprising all found CFIs of an according type in a software project.
This aids us in examining several software projects to gain sufficient and sensible data about commit-feature interactions.
The created reports are also evaluated in the VaRA-Tool-Suite, which offers support to process and display statstics of the generated data. \\
\fi 
