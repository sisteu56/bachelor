%************************************************
\section{Introduction}\label{ch:introduction}
%************************************************

Features play an important part in modern programming, which shows in program paradigms such as feature-oriented programming,
where they are used to implement specific functionalities, can be activated or deactivated and thus add configurability to software systems.
As commits are used to contribute new source-code to a repository,
it follows that commits can structurally interact with features by introducing source-code implementing their functionality. 
Furthermore, seemingly unrelated code, that was changed or added by commits, might influence soure-code of a feature through dataflow.
We aim to give insights into these topics by investigating \nameref{ch:example_chapter} (CFIs).

Within a program, there exist many different abstract entities, such as commits and features, each serving different responsibilites.
For a better understanding of them, it is advantageous to know whether they interact with or among each other.
In the context of CFIs, bugs occuring in a feature could be linked to the latest commits affecting data of said feature, and consequently the authors responsible for these bugs.
Especially for dataflow spanning over multiple files and many lines of code, it might be difficult to determine the respective interactions by studying the program yourself.
To enable an automatic detection of these interactions and allow for a complete overview of them inside a software project, Sattler et al.~\cite{sattler2023thesis} created the interaction analysis tool VaRA.

In a prior study \citet{sattler2023seal} used VaRA to examine interactions between commits showing that research on this topic can be applied to improve many aspects of software development.
For example, their research allowed for a deeper understanding of interactions between developers by linking commit interactions to their respective authors.

We pick up and extend on their research by investigating interactions between commits and features.
For this, we introduce the concept of structural and dataflow-based CFIs.
Structural CFIs occur inside a program when there is overlap between source-code changed by a commit and source-code constituting a feature.
This overlap implies that the commit changed or implemented functionality of the feature, meaning that structural CFIs can produce data on feature development.
Dataflow-based CFIs on the other hand, occur when there exists dataflow from the code representations of commits to that of features.
Thus, commits of these interactions introduced changes in the program affecting data, which is later used inside a feature.
CFIs based on dataflow examine the program on a deeper layer, allowing us to detect additional interactions missed by structural CFIs.
By combining dataflow-based and structural interactions, we can specify whether dataflow stems from outside or from inside the code constituting a feature, thus producing more informative data.

\subsection{Goal of this Thesis}\label{sec:thesis_goal}

The primary focus of this thesis is to gain an overview of how commits interact with features in software projects. 
Our goal is to lay basic groundwork regarding this
subject, while leaving more detailed questions to future research. As previously mentioned,
we investigate two types of commit-feature interactions, namely structural and dataflow-based interactions. While
using both types separately can already answer many research questions, we also show
applications utilizing a combination of both.
We aim to reveal insights about the development process of features and usage of commits therein 
with the help of structural CFIs and high-level repository information.
We also investigate to what extend there exist CFIs
through dataflow that cannot be discovered with a purely syntactical analysis.

\subsection{Overview}\label{sec:overview}

In \nameref{ch:related_work}, we summarize previous studies investigating interactions inbetween commits and inbetween features to motivate that interactions between them is a topic worthy of study.
The \nameref{ch:background} chapter serves as an introduction to the concepts of code regions and interaction analysis, which are necessary to properly define structural and dataflow-based CFIs.
Their definition then takes place in the \nameref{ch:example_chapter} chapter, where we thouroughly dicuss their meaning in software projects as well as our \nameref{ch:implementation} of their detection in VaRA.
The research questions of this paper are established in the \nameref{ch:methodology} chapter, in which we also explain how we plan on investigating them.
Furhtermore, methodology contains the sections \nameref{sec:expectations} and \nameref{sec:threats}.
To finish the proposal, we summarize our proposed research questions and their planned evaluation in the \nameref{ch:conclusion}.
