%************************************************
\section*{Conclusion}\label{ch:conclusion}
%************************************************

In this work we research the main properties of structural and dataflow-based commit-feature interactions.
Using high-repository information about commits and their authors as well as a combination of both types of interactions allows us to gain additional knowledge on their properties.
To investigate these properties we examine several small software projects revolving around compression, such as xz.
For this, we create reports containing all found structural and dataflow-based commit-feature interactions of the to be  examined projects.
The collected data is then processed to facilitate performing calculations with it and displaying it graphically.
Structural interactions and the injection of author information within them can be utilized to provide insights into feature development and usage of commits therin.
This includes showing how often commits implement more than one feature and how strongly correlated the size of a feature and the number of commits used to implement it are.
Furthermore we calculate the average number of authors that implement a feature in a project.
Dataflow interactions are used to unveil interactions between features and commits that cannot be discovered through a purely structural analysis.
Seeing how common they really are could encourage programmers to be more aware of them.
To gain some insight on this, we measure what fraction of commits contributing code to the project affect features through dataflow.
This can also improve our understanding on which impact new commits have on features, as it gives us an estimate of how likely new commits are to influence the data of a feature. 



\iffalse
Research involving interactions inbetween features and commits has shown that investigating interactions between program entities is a topic worthy of study.
For example, Sattler et al used dataflow commit interactions to allow for a more detailed understanding of author interactions in software projects. 
Commit interactions also make it possible to identify seemingly insignificant changes that have a central impact on the program.
Kolesnikov et al provides further indication for the wide range of subjects interactions can be used for.
Particularly, they argued that control-flow interactions between features can help predict performance interactions between them. 

We extend VaRA to implement the detection of structural and dataflow-based commit-feature interactions.
As we want to investigate several projects, we create reports containing all found interactions for them.
These reports are created inside the VaRA-Tool-Suite and contain either all structural or dataflow-based interactions of a software project.
The VaRA-Tool-Suite also allows use to work with the collected data and enrich it with high-repository information, such as information on which author a commit belongs to.
\fi
