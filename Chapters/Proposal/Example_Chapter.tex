\subsection*{Example Chapter}\label{sec:example_chapter}

The detection of structural as well as dataflow-based commit-feature interactions is implemented in VaRA \cite{VaRA2023}.
VaRA offers two main functionalities for this. 
The first one is the detection of feature- and commit-code, which is accomplished by being able to receive all commits and features an llvm-IR instruction belongs to.
Thus we can collect all structural commit-feature interactions by iterating over all instructions in the code space.
Inside an instruction we save every combination of commits and features as a CFI.
It follows that, in order for an instruction to have a single interaction, it needs to be part of at least one commit as well as one feature.
For each structural CFI we also save the amount of instructions said cfi occurs in. 
This is accomplished by incrementing the instruction counter if we happen to encounter a duplicate CFI. 
VaRA is also able to track taints of values along program flow, where taints essentially carry information on which commits have previously affected that specific value.
Similarly to structural interactions, dataflow-based commit-feature interactions are iteratively collected on instruction level.
In this case we speak of an dataflow interaction when an instruction both has a commit taint and belongs to a feature.
Consequently this instruction uses a value that was changed by a commit earlier in the program while stemming from code constituting a feature.
For our research we will examine numerous software projects to get a wide range of reference data, as commit-feature interactions could potentially vary greatly between different code spaces.
Accordingly, the VaRA-Tool-Suite was extended making it possible to generate a report comprising all found CFIs of an according type in a software project.
This will aid us in examining several software projects to gain sufficient and sensible data about commit-feature interactions.
The created reports will also be evaluated in the VaRA-Tool-Suite, which offers support to process and display statstics of the generated data. \\
