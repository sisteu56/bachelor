%************************************************
\section*{Background}\label{ch:background}
%************************************************

\subsection*{Interaction Analysis}

Essential to the research conducted in this paper is the interaction analysis introduced by \citet{sattler2023seal}.
Their interaction analysis tool SEAL is implemented on top of LLVM and PhASAR.
Their approach will be shortly discussed here, however we advice their paper for a more thorough explanation. \\
The first step of their approach is to annonate code by mapping information to the compiler's intermediate representation (IR).
This information is added to the LLVM-IR instructions during the construction of the IR.
SEAL focuses on commit-information, which encompasses the commit's hash and its respective repository.
We say that the commit of an instruction is the commit that last changed the source-code line the instruction stems from.
Determining the commit that last changed a source code line is achieved by accessing repository meta-data. \\
The second analysis step involves the actual computation of the interactions.
For this, Sattler et al. implemented a special, inter-procedural taint analysis.
It's able to track data flows between the instructions of a given target program.
Such dataflow interactions occur when an instruction uses data as its input that was changed or allocated by an instruction earlier in the program.
Commit interactions can be determined through pairs of instructions that interact with eachother through dataflow.
This is accomplished by lifting instructions according to their respective commits made possible by an instruction's mapped commit information.

\subsection*{Definitions}

\ \ \textbf{Definition 1.} \textbf{Commits} are used within a version control system to introduce the latest source code changes to its respective repository.
Inside a repository revision, a commit encompasses all source code lines that were last changed by it.
The sum of all source code lines belonging to a commit within a revision is called a \textbf{commit region}. \\

\textbf{Definition 2.} \textbf{Commit taints} are used to track data along program flow essentially carrying information on which commit regions have previously affected this data.
An instruction is tainted by a commit if it uses data, that has been changed or allocated inside a commit region, as input. \\

\textbf{Definition 3.} In general, \textbf{features} are parts of a program implementing specific functionality.
In this work we focus on features that are modelled with the help of configuration variables.
Configuration variables, also called feature variables, decide whether source code, performing a feature's intended functionality, gets executed.
The sum of all source code lines whose execution depends on a feature variable is called a \textbf{feature region}. \\

We refine the computation of code regions, that was introduced by Settler et al., by defining the computation of commit and feature regions.
This allows allows a precise definition of commit-feature interactions through interactions between commit and feature regions. \\

\textbf{Definition 4.} With computeCommitRegions(revs), we compute all commit regions for the specified revisions in revs $\subseteq$ R, 
by computing the revision-specific commit regions for each program revision $\text{p}^{\text{rev}}$ with rev $\in$ revs.
\begin{center} computeCommitRegions(revs) = $\bigcup\limits_{\text{rev} \in \text{revs}}$ $\bigcup\limits_{\text{p} \in \text{rev}}$ $\bigcup\limits_{\text{f} \in \text{p}}$ 
	$\text{computeCR}_{\text{Tag}}\text{(f,t=Commit)}$ 
\end{center}

\textbf{Definition 5.} With computeFeatureRegions(revs), we compute all feature regions for the specified revisions in revs $\subseteq$ R, 
by computing the revision-specific feature regions for each program revision $\text{p}^{\text{rev}}$ with rev $\in$ revs.
\begin{center} computeFeatureRegions(revs) = $\bigcup\limits_{\text{rev} \in \text{revs}}$ $\bigcup\limits_{\text{p} \in \text{rev}}$ $\bigcup\limits_{\text{f} \in \text{p}}$ 
	$\text{computeCR}_{\text{Tag}}\text{(f,t=Feature)}$ 
\end{center}

