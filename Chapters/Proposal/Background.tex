%************************************************
\section*{Background}\label{ch:background}
%************************************************

In this section, we summarize previous research on the topic of code regions and interaction analysis.
Thereby we give defintions of terms and dicuss concepts that are fundamental to our research.
In the next chapter, we will use the introduced definitions and concepts to explain and investigate structural as well as dataflow-based commit-feature interactions.

\subsection*{Code Regions}\label{ch:code_regions}
Software Programs consist of source code lines that are translated into an intermediate representation (IR) upon compilation.
IR accurately represents the source-code information and is utilized in many code improvement and transformation techniqes such as code-optimization.
Furthermore, static program analyses are conducted on top of IR or some sort of representation using IR. 
Code regions are comprised of IR instructions that are directly consecutive in their control-flow. 
They are used to represent abstract entities of a software project, such as commits and features.
For this, each code region carries some kind of variability information, detailing its meaning.
It should be noted that there can be several code regions with the same meaning scattered across the program.
We dicuss two kinds of variablity information, namely commit and feature variability information allowing us to define two separate code regions. \\
Commits are used within a version control system to represent the latest source code changes in its respective repository.
Inside a repository revision, a commit encompasses all source code lines that were last changed by it. 

\theoremstyle{definition}
\newtheorem{definition}{Definition}
\begin{definition}\label{def:commit_regions}
	The sum of all consecutive instructions, that stem from source code lines belonging to a commit, is called a \emph{commit region}.
\end{definition}

As the source code lines changed by a commit are not necessarily contigious,
a commit can have many commit regions inside a program. 
To properly address the entire set of these commit regions within a program, we introduce the term regions of a commit.
We say that the \emph{regions of a commit} are the set of all commit regions that stem from source code lines of the commit. \\

In general, features are program parts implementing specific functionality.
In this work we focus on features that are modelled with the help of configuration variables that specify whether the functionality of a feature should be active or not.
Inside a program, configuration variables decide whether instructions, performing a feature's intended functionality, get executed. 
Detecting these control-flow dependencies is achieved by deploying extended static taint analyses, such as Lotrack.

\begin{definition}\label{def:feature_regions}
	The sum of all consecutive instructions, whose execution depends on a configuration variable belonging to a feature, is called a \emph{feature region}. 
\end{definition}

Similarly to commits, features can have many feature regions inside a program,
as the instructions implementing their functionalities can be scattered across the program.
Here, we also say that the \emph{regions of a feature} are the set of all feature regions that stem from a feature's configuration variable. 

\subsection*{Interaction Analysis}\label{sec:interaction_analysis}

In the previous chapter, we have already shown how different abstract entities of a software project, such as features and commits, can be assigned concrete representations within a program.
We can use interaction analyses to infer interactions between them by computing interactions between their concrete representations.
This can improve our understanding of them and give insights on how they are used inside software projects. 
An important component of computing interactions between concrete representations is the concept of interaction relations between code regions introduced by \citet{sattler2023thesis}.
As we investigate structural and dataflow-based interactions, we make use of structural $\circledcirc$ and dataflow $\rightsquigarrow$ interaction relations in this work. \\
The structural interaction relation between two code regions $r_1$ and $r_2$ is defined as follows:

\begin{definition}\label{def:structural_relation}
	The interaction relation $r_1 \circledcirc r_2$ evaluates to true if at least one instruction that is part of $r_1$
	is also part of $r_2$.
\end{definition}

The dataflow interaction relation between two code regions $r_1$ and $r_2$ is defined as:

\begin{definition}\label{def:dataflow_relation}
	The interaction relation $r_1 \rightsquigarrow r_2$ evaluates to true if the data produced by at least one instruction i that is part of $r_1$
	flows as input to an instruction $i'$ that is part of $r_2$.
\end{definition}

Essential to the research conducted in this paper is the interaction analysis introduced by \citet{sattler2023thesis}.
Their interaction analysis tool VaRA is implemented on top of LLVM and PhASAR.
VaRA is used to determine dataflow interactions between commits by computing interactions between their respective commit regions.
This computation is carried out using the dataflow interaction relation $\rightsquigarrow$.
Their approach for this will be shortly discussed here, however we advice their paper for a more thorough explanation. \\
The first step of their approach is to annonate code by mapping information about its regions to the compiler's intermediate representation.
This information is added to the LLVM-IR instructions during the construction of the IR.
VaRA focuses on commit regions, which contain information about the commit's hash and its respective repository.
The commit region of an instruction is extracted from the commit that last changed the source code line the instruction stems from.
Determining said commit is accomplished by accessing repository meta-data. \\
The second analysis step involves the actual computation of the interactions.
For this, \citet{sattler2023thesis} implemented a special, inter-procedural taint analysis.
It's able to track data flows between the commit regions of a given target program.
For this, information about which commit region affected it, is mapped onto data and tracked along program flow.
In this context, we would like to introduce the concept of taints, specifically commit taints.
Taints are used to to give information about which code regions have affected an instruction through dataflow.
For example, an instruction is tainted by a commit if its taint stems from a commit region.
It follows, that two commits, with their respective commit regions $r_1$ and $r_2$, interact through dataflow, when an instruction is part of one's commit region, $r_2$, while being tainted by $r_1$.
Consequently data produced by the commit region $r_1$ flows as input to an instruction within the commit region $r_2$, matching the dataflow-based interaction relation as defined in definition~\ref{def:dataflow_relation}.

