%************************************************
\section*{Methodology}\label{ch:methodology}
%************************************************

This chapter describes the methodology of the thesis core evaluation.

\subsection*{Research Questions}\label{sec:research_questions}

The primary focus of this thesis is to gain an overview of how Commits interact with Features in Software Projects.
Specifically our goal is to lay basic groundwork regarding this subject, while leaving more detailed questions to future research.
As previously mentioned we investigate two types, strutctural and dataflow-based Commit-Feature Interactions.
While using both types separately can already answer many research questions, we will also show applications utilizing a combination of both.

\textbf{Structural Interactions} 

We define structural Commit-Feature Interactions as the occurence of syntactical overlap between commits and features.
The syntax of a commit is comprised of commit code, namely code that was last changed by said commit.
Feature syntax is defined in the same way as being comprised of feature code. 
Identifying feature code is more complex however, as it is any code whose execution depends on a feature being active or directly uses some feature variables. 
Now, we say that a commit and a feature interact structurally when commit code is part of feature code and vice versa. 

\textbf{RQ1: What are the characteristics of structural Commit-Feature Interactions?}

How many commits does a feature interact with structurally? In how many Features can we find a certain commit?
These and many more questions surround the development process of Features and Software Projects in general, for which we aim to present the statistical evidence found in our research.
With the acquired the data, we can answer many interesting assumptions about said development process.
For example one might assume that, because they generally change a specific functionality, commits mostly are feature-specific, e.g. only affect the code of a single feature.
This can give further insight on what purpose commits serve in a Software Developer's work.

\textbf{Dataflow-based Interactions} 

We define dataflow-based Commit-Feature Interactions as data that was changed by a commit being accessed by a feature later in the program flow and vice versa.
For programmers it can be difficult to be aware of these dependencies as they might span over large spaces of code and several files.
We provide software that can automatically discover these, which can aid a programmer's ability to find errors and remove bugs.
For example for a bug occuring in a certain feature, we can unveil commits that might have caused it, while not actually being part of that feature.

\textbf{RQ2: How do Commits interact with Features through Dataflow?}

We want to provide insight on the prevelence of dataflow-based Commit-Feature Interactions in Software Projects.
Commits that constitute parts of feature code are obviously likely to interact with features through dataflow.
Investigating wether this is also true for commits that aren't part of a feature will show us how interconnected features and separate commits are.
The produced data will also help shine light on the question of how isolated or respectively dependent from features are on other parts of a program.

\subsection*{Operationalization}\label{sec:operationalization}

In this section, present how you want to evaluate your thesis.

\subsection*{Expectations}\label{sec:expectations}

In this section, discuss the results you expect to get from your evaluation.

\subsection*{Threats to Validity}\label{sec:threats}

In this section, discuss the threats to internal and external validity you have to be aware of during the evaluation.
