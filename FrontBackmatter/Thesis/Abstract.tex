%*******************************************************
% Abstract
%*******************************************************
%\renewcommand{\abstractname}{Abstract}
\pdfbookmark[1]{Abstract}{Abstract}
% \addcontentsline{toc}{chapter}{\tocEntry{Abstract}}
\begingroup
\let\clearpage\relax
\let\cleardoublepage\relax
\let\cleardoublepage\relax

\chapter*{Abstract}

- in software engineering, features play a pivotal role in implementing functionality of applications and adding configurability to them \\
- their code is maintained in git repositories, where commits are used to introduce the latest source-code changes, thus gradually building the overall application and its features \\ % commits are an incremental unit of development
- to allow for a deeper understanding of commits and features as well as their interplay, we connect both entities in the form of commit-feature interactions (CFIs) \\
- for this, we extend the interaction analysis tool VaRA to implement the detection of structural and dataflow-based CFIs \\
- while the former provides information about the direct involvement of commits in developing features, the latter can uncover seemingly unrelated commits still influencing the functionality of features \\
- as previous research has focused exclusively on interactions either among commits or among features, we are first to study interactions between the two entities \\
- specifically, we show use-cases of investigating CFIs including insights into the development process of features and usage of commits therein \\
- this enabled us, for instance, to confirm a derived best practice surrounding commits in feature development—namely, that commits should only change a single feature \\ 
- furthermore, our dataflow analysis 

- with this thesis, we accomplished an initial overview of CFIs, their applications and dependencies between the two introduced types \\
- 

\vfill

\endgroup

\vfill
