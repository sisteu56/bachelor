%*******************************************************
% Abstract
%*******************************************************
%\renewcommand{\abstractname}{Abstract}
\pdfbookmark[1]{Abstract}{Abstract}
% \addcontentsline{toc}{chapter}{\tocEntry{Abstract}}
\begingroup
\let\clearpage\relax
\let\cleardoublepage\relax
\let\cleardoublepage\relax

\chapter*{Abstract}

In software engineering, features play a central role in implementing functionality of applications and making them configurable. 
Their code is maintained in Git repositories, where commits are used to introduce the latest source-code changes, thus gradually building the overall application and its features. 
To allow for a deeper understanding of commits and features as well as their interplay, we connect both entities in the form of commit-feature interactions (CFIs).
For this, we extend the interaction analysis tool VaRA to implement the detection of structural and dataflow-based CFIs.
While the former provides information about the direct involvement of commits in implementing features, the latter can uncover seemingly unrelated commits still influencing their functionality.

Similarly to prior research~\cite{michelon2021lifecycle}, we use structural CFIs to investigate the development process of features and usage of commits therin.
Instead of focusing on the number of features a commit changes, we study the responsiblities a commit deals with in relation to features giving us a more accurate reflection of a commit's purpose in feature development.
Specifically, we argue that a commit still deals with a single feature-related responsibility when it changes code of a nested feature or code implementing the interplay of several features.
We derive a best practice for commits, namely that commits should usually deal with only one feature-related responsibility, which we find to be enforced for the majority of commits (>69\%).% in our examined projects. 

Furthermore, we choose a novel approach to fully assess the interplay between commits and features by being the first to study dataflow interactions between the two entities. % asses or understand
We find that a substantial fraction of a project's commits interact with features through dataflow, although the respective fraction varies strongly between the examined projects ($\math{11-37}$\%).
% to assess additional interactions dataflow analysis can reveal, we check whether a commit and a feature interacting through dataflow also structurally interact with each other
% here, we determine that commits and features interact on both types very often
% the reason for this is that structural CFIs heavily coincide with dataflow-based CFIs 

To assess whether a dataflow interaction occurs between on first sight unrelated commits and features, we determine whether the commit is located inside or outside 








We determine what additional information a dataflow analysis can add by differentiating between interactions, which we either did or did not already discover with our structural analysis. % on both commit and feature side by 
Regarding all commits interacting with features, our dataflow analysis revealed a substantial fraction of them exclusively ($\math{20-56}$\%) and additional interactions for surprisingly many commits ($\math{20-60}$\%)\footnote{percentages depend on the examined project}.
For the majority of the investigated features we also find that most commits interact with them only through dataflow.
Thus, we show that our dataflow analysis is an essential part of gaining a complete overview of the interplay between commits and features in the examined software projects.
% assess the umfang of commits with dataflow interactions
% abgleichen das mit dem set an interaktionen die wir schon mit unserer strukturellen analyse gefunden haben, da eine strukturelle interaktionen zwischen einem commit und feature bedingt, dass sie ebenfalls fast immer (>90%) auch über datenfluss interagieren
% so bestimmen wir wie viel neue Informationen eine datenfluss analyse bringen kann
% wir stellen fest, dass die meisten Features mehrheitlich von Commits ausschließlich über datenfluss beeinflussen werden
% 
% one major goal was to determine the additional interacting commits that can be discovered for features through a dataflow analysis 

% In this work, we also illustrate how CFIs can be applied to investigate an alternative albeit related form of interaction: author-feature interactions, achieved by mapping commits to their respective authors.


\iffalse 

As previous studies have focused exclusively on interactions either between commits or between features, our work focuses on interactions between commits and features. % we delve into research topics that neither can answer on their own,

%Sehr unklar
% Was ist deine research topic
We address research topics that neither can answer on their own, including insights into the development process of features and usage of commits in it. % , in particular we show use cases for the investigation of CFIs,

We use our method to evaluate best practices in software/feature development, e.g., that commits should only change a single feature.

In order to study this, we did X

We found that this is a good idea because of results X Y Z

This enabled us, for instance, to confirm a derived best practice surrounding commits in feature development - namely, that commits should only change a single feature.
Furthermore, we find our employed dataflow analysis to be an essential part of gaining a complete overview of the interactions between commits and features in a project, as the majority of them take place only at dataflow level.
%It should be noted that a substantial amount of dataflow occurs within or between features, and does not stem from other parts of a program.
%That is, because commits constituting code of features have an almost certain chance of interacting with them via dataflow as well as a significantly increased probability of influencing other features. % this designates...

% RQ nicht definiert!
%As some examined projects proved to be unfit for investigation of author-feature interactions, we were not able to conclusively answer our research questions here.

% Conclusion
%Nevertheless, we are convinced that this form of interaction promises to improve the understanding of developers on their interplay with features.
% based on the applications we have worked out

%With this thesis, we accomplished an initial overview of commit-feature interactions, their use-cases as well as further applications and utilized them to determine dependencies between the involved entities. % the two involved entities

\fi 

\vfill

\endgroup

\vfill
