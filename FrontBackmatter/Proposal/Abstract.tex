%*******************************************************
% Abstract
%*******************************************************
%\renewcommand{\abstractname}{Abstract}
\pdfbookmark[1]{Abstract}{Abstract}
% \addcontentsline{toc}{chapter}{\tocEntry{Abstract}}
\begingroup
\let\clearpage\relax
\let\cleardoublepage\relax
\let\cleardoublepage\relax

\section*{Presentation Abstract}\label{ch:abstract}

In software engineering, features play a pivotal role in implementing functionality of software applications.
There, commits are used to introduce new source code and thus gradually build the overall application and its features.
In this paper we connect both entities of software projects in the form of commit-feature interactions (CFIs).
For this, the interaction analysis tool VaRA is extended to implement the detection of structural as well as dataflow-based interactions between commits and features.
We carry out a comprehensive investigation of CFIs in software projects to provide data-driven conclusions on yet unexplored research areas related to features.
These research areas include feature development and the effect of new changes on feature functionality.
By analyzing CFIs, we shine light on the dependencies between commits and features, enabling developers to be more aware of them and thus reducing potential errors during software development.

\iffalse

Problem: \\
- features are an important part of software engineering \\
- as with any other part of a software project, commits are used to develop these features \\
- No hands-on properties of feature development inside a program \\
- Unknown how common it is for new changes to affect features in a program \\
Why it is a problem: \\
- Nowadays data is used to fix/improve many programming aspects, the same can be done for feature development \\
- Being unaware of certain dependencies makes it harder to find errors related to them \\
Startling sentence: \\
Investigating commit-feature interactions (CFI) gives important insights into feature development and can reveil how seemingly unrelated changes affect the functionality of a feature. \\
Implication of the statement: \\
- Doing research on CFIs can provide data that lets us draw interesting conclusions related to the problems which might help solve them \\
Short summary of the contents in English\dots a great guide by
Kent Beck how to write good abstracts can be found here:
\begin{center}
\url{https://plg.uwaterloo.ca/~migod/research/beckOOPSLA.html}
\end{center}

\fi

\vfill

\endgroup

\vfill
