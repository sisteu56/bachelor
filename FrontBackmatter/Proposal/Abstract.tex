%*******************************************************
% Abstract
%*******************************************************
%\renewcommand{\abstractname}{Abstract}
\pdfbookmark[1]{Abstract}{Abstract}
% \addcontentsline{toc}{chapter}{\tocEntry{Abstract}}
\begingroup
\let\clearpage\relax
\let\cleardoublepage\relax
\let\cleardoublepage\relax

\section*{Presentation Abstract}\label{ch:abstract}

In software engineering, features play a pivotal role in implementing functionality of applications and adding configurability to them.
Their code is maintained in git repositories, where commits are used to introduce the latest source-code changes, thus gradually building the overall application and its features.
To allow for a deeper understanding of commits and features as well as their interplay, we connect both entities in the form of commit-feature interactions (CFIs). 
For this, we extend the interaction analysis tool VaRA to implement the detection of structural and dataflow-based CFIs.
As previous research has focused exclusively on interactions either among commits or among features, we are first to study interactions between the two entities.
Specifically, our goal is to provide data-driven conclusions on research areas that neither could give on their own.
These research areas include feature development, where we examine the number of commits and authors needed to implement a feature in relation to its size. 
We also investigate how commits are used during the development of features to check whether their usage follows best practices revolving around them. 
Furthermore, we examine dataflow-based CFIs to reveal interactions between seemingly unrelated commits and features that cannot be detected with a purely structural analysis.
This allows us to determine how likely commits without any source-code contributions to features are to affect them through dataflow.
By analyzing CFIs, we shine light on the dependencies between commits and features inside software projects, enabling developers to be more aware of them and thus reducing potential errors during software development.

\iffalse

Problem: \\
- features are an important part of software engineering \\
- as with any other part of a software project, commits are used to develop these features \\
- No hands-on properties of feature development inside a program \\
- Unknown how common it is for new changes to affect features in a program \\
Why it is a problem: \\
- Nowadays data is used to fix/improve many programming aspects, the same can be done for feature development \\
- Being unaware of certain dependencies makes it harder to find errors related to them \\
Startling sentence: \\
Investigating commit-feature interactions (CFI) gives important insights into feature development and can reveil how seemingly unrelated changes affect the functionality of a feature. \\
Implication of the statement: \\
- Doing research on CFIs can provide data that lets us draw interesting conclusions related to the problems which might help solve them \\
Short summary of the contents in English\dots a great guide by
Kent Beck how to write good abstracts can be found here:
\begin{center}
\url{https://plg.uwaterloo.ca/~migod/research/beckOOPSLA.html}
\end{center}

\fi

\vfill

\endgroup

\vfill
